% !TEX root = ./Matematica discreta.tex

\chapter{Cenni di Teoria degli Insiemi}\label{chapt:Cenni_Teoria_Insiemi}
Il concetto di Insieme è primitivo, ovvero non può essere definito senza coinvolgere altri concetti che a loro volta non possono essere definiti. Com'è possibile definire un insieme allora? Vi sono vari modi:
\begin{enumerate}
    \item elencare tutti i suoi oggetti. Seguono alcuni esempi:
        \begin{enumerate}[label=\roman*.]
            \item \(A = \set{0, -1, \sqrt{3}, 6, 28}\);
            \item \(B = \set{a, x, b, 3}\);
            \item \(C = \set{2, +, *, :, t}\);
            \item \(\N = \set{0, 1, 2, 3, \dotsc}\), ovvero l'insieme dei numeri naturali;
            \item \(\Z = \set{\dotsc, -2, -1, 0, 1, 2, \dotsc}\), ovvero l'insieme dei numeri relativi.
        \end{enumerate}
    \item esprimere una proprietà caratteristica, ovvero una proprietà che venga verificata da tutti e soli gli oggetti facenti parte di quell'insieme. In simboli:
    \[
        A = \set{\isdiv{x}{P}(x)}.
    \]
    Seguono ora degli esempi:
        \begin{enumerate}[label=\roman*.]
            \item \(D = \set{\isdiv{x}{x} \text{ è una lettera dell'alfabeto italiano}}\);
            \item \(E = \set{x \mid \forall n \in \N \quad x = 2n}\), l'insieme dei numeri naturali pari;
            \item \(\N_* = \set{\isdiv{n}{n} \in \N \setminus \{0}\}\);
            \item \(\N = \set{\isdiv{x}{x} \in \Z_+}\).
        \end{enumerate}
    \item rappresentare graficamente l'insieme tramite diagramma di Eulero-Venn:
    % !TEX root = ../Matematica discreta.tex

\begin{figure}[h]
\centering
    \begin{tikzpicture}[
            group/.style={ellipse, draw, minimum height=50pt, minimum width=70pt, label=above:#1},
            my dot/.style={circle, fill, minimum width=2.5pt, label=#1, inner sep=0pt}
        ]

        \node (a) [right=30pt, my dot=right:\(a\)] {};
        \node (b) [yshift=16pt, xshift=-20pt, below=of a, my dot=right:\(b\)] {};
        \node (x) [yshift=16pt, xshift=+16pt, below=of b, my dot=right:\(x\)] {};
        \node (3) [yshift=16pt, xshift=-25pt, below=of x, my dot=right:\(3\)] {};

        \node [fit = (a) (b) (x) (3), group=\(B\)] {};
    \end{tikzpicture}
\end{figure}
\end{enumerate}

Dunque un insieme è formato da oggetti. Se \(A\) è un insieme, la circostanza che un oggetto \(a\) faccia parte degli oggetti che costituiscono l'insieme \(A\) si esprime dicendo che ``\(a\) appartiene ad \(A\)'' oppure che ``\(a\) è elemento di \(A\)''. In simboli si esprime tramite l'utilizzo di ``\(\in\)'', ovvero il simbolo di appartenenza. Scriveremo dunque:
\[
    a \in A, \qquad A \ni a;
\]
l'oggetto \(a\) si dice \emph{elemento} dell'insieme \(A\). È possibile scrivere proposizioni che hanno un senso e un valore di verità utilizzando il simbolo di appartenenza, ad esempio: \(7 \in \N\) (\true[l]) e \(\pi \in \N\) (\false[l]).

Se si vuole esprimere in simboli la circostanza che un oggetto \(a\) non appartenga ad \(A\) si utilizzerà ``\(\notin\)'':
\[
    a \notin A, \qquad A \centernot\ni a.
\]
In altre parole \(\neg(a \in A)\) si scrive \(a \notin A\), per esempio: \(\pi \notin \N\).


\section{Numeri razionali}\label{susec:numeri_razionali}
Abbiamo già introdotto gli insiemi \(\N\) e \(\Z\), introduciamo ora l'insieme dei numeri razionali: \(\Q\). Questo è l'insieme di tutte le frazioni, ovvero l'insieme di tutti i numeri decimali con cifre periodiche. \(\Q\) è definito nel seguente modo:
\[
    \Q = \set{\frac{p}{q} \;\middle|\; (p,q \in \Z) \land (q \neq 0)}.
\]

Siano \(p/q, r/s \in \Q\) con \(q,s \neq 0\). Le due frazioni saranno uguali quando:
\[
    \frac{p}{q} = \frac{r}{s}
    \iff
    (\exists h \in \Z \tc (r = hp) \land (s = hq)) \lor (\exists k \in \Z \tc (p = kr) \land (q = ks)).
\]
Ad esempio, \(1/2 = 2/4\), perché \(2 = 2 \cdot 1\) e \(4 = 2 \cdot 2\); in questo caso \(h = 2\).

Presa la frazione \(p/q\), chiameremo \(p\) \textbf{numeratore} e \(q\) \textbf{denominatore}. Considerata la frazione \(-(3/4)\), essa potrà essere scritta come \((-3)/4\) oppure come \(3/(-4)\).


\subsection{Decimali e Frazioni}\label{subsec:decimali_frazioni}
Come detto in precedenza, i numeri razionali possono essere espressi in due modi diversi, o come frazioni o come numeri decimali. Questo vuol dire che è possibile passare da una forma all'altra, vediamo come.

Per passare da frazione a decimale il procedimento risulta essere molto semplice, è necessario dividere il numeratore per il denominatore, ad esempio:
\[
    \frac{10}{3} = 3.\bar{3},
    \qquad
    \frac{1}{2} = 0.5\bar{0}
\]
Nel caso in cui non sia presente alcun decimale periodico, vuol dire che viene omesso \(\bar{0}\), infatti \(0.5 = 0.5\bar{0}\).

Per passare da decimale a frazione il procedimento è leggermente più complesso. Siano \(c \in \Z\), \(a_1, \dotsc, a_h \in \N\) e \(b_1, \dotsc, b_k \in \N\); si consideri il generico numero decimale \(q\):
\[
    q = c.a_1 a_2 \dotsc a_h \overline{b_1 b_2 \dotsc b_k}.
\]
Si ha che
\[
    c.a_1 a_2 \dotsc a_h \overline{b_1 b_2 \dotsc b_k} = \frac{c a_1 \dotsc a_h}{10^h} + \frac{b_1 \dotsc b_k}{9k \cdot 10^h}.
\]
La formula precedente permette di convertire un numero decimale in una frazione, ad esempio:
\[
    0.\bar{9} = \frac{0}{10^0} + \frac{9}{9 \cdot 10^0} = 0 + \frac{9}{9} = 0 + 1 = 1.
\]
In generale si ha che
\[
    a.\bar{9} = a + 0.\bar{9} = a + 1.
\]
Un altro esempio è il seguente:
\[
    10.01\overline{211} = \frac{1001}{10^2} + \frac{211}{999 \cdot 10^2}
                        = \frac{1001 \cdot 999 + 211}{99 \cdot 900}
                        = \frac{1,000,21\cancel{0}}{99,90\cancel{0}}
                        = \frac{100,021}{9,990}
\]


\section{Numeri reali}
Introduciamo ora l'insieme dei numeri reali. Questi sono numeri razionali e irrazionali, che sono numeri decimali non periodici. Alcuni esempi di numeri irrazionali sono
\begin{alignat*}{1}
        \sqrt{2} &= 1.414213 \dotsc \\
        \sqrt{5} &= 2.236067 \dotsc \\
             \pi &= 3.141592 \dotsc
\end{alignat*}
L'insieme si denota con \(\R\)


\section{Insiemi numerici}\label{sec:insiemi_numerici}
\begin{definition}[sottoinsieme improprio]\label{def:sottoinsieme_improprio}
    Siano \(A,B\) due insiemi. Si dice che \(A\) è un sottoinsieme di \(B\) oppure che \(A\) è contenuto in \(B\) o ancora che \(A\) è incluso in \(B\) se ogni elemento di \(A\) è anche elemento di \(B\). Si scriverà \(A \subseteq B\).
\end{definition}

\begin{examples}
    Seguono alcuni esempi di insiemi e sottoinsiemi:
        \begin{enumerate}
            \item \(\set{3, -2, 5} \subseteq \Z\);
            \item \(\N \subseteq \Z \subseteq \Q \subseteq \R\);
            \item Considerati \(A = \set{7, a, -1, *}\) e \(B = \set{-3, 7, a, -1, *, \cdot}\), si ha che \(A \subseteq B\);
        \end{enumerate}
\end{examples}

\begin{remark}
    Appartenenza e inclusione sono diversi: \(\in \mathop{\neq} \subseteq\), infatti scrivere \(3 \subseteq \N\) è errato, bisognerebbe scrivere \(3 \in \N\) oppure \(\{3\} \subseteq \N\).
\end{remark}


\begin{definition}[insiemi uguali]\label{def:insiemi_uguali}
    Siano \(A,B\) due insiemi. Si dice che \(A = B\) se e solo se i due insiemi hanno gli stessi elementi, ovvero ogni elemento di \(A\) è anche elemento di \(B\) e ogni elemento di \(B\) è anche elemento di \(A\).
\end{definition}

\begin{remark}
    Si avrà che \(A = B\) quando \((A \subseteq B) \land (B \subseteq A)\). In simboli scriveremo:
    \[
        A = B \iff \left[(\forall x \in A) (x \in B)\right] \lor \left[(\forall y \in B) (y \in A)\right].
    \]
\end{remark}

\begin{remark}
    Dire che \(A\) non è sottoinsieme di \(B\), significa dire, in simboli, che:
    \[
        \neg(A \subseteq B) \iff \neg \left[(\forall x \in A) (x \in B)\right]
                            \iff (\exists x \in A) (x \notin B)
    \]
\end{remark}

\begin{examples}
Seguono due esempi:
    \begin{enumerate}
        \item \(\Z \nsubseteq \N\) perché ad esempio \(\exists {-}4 \in \Z \tc {-}4 \notin \N\);
        \item Considerati \(A = \set{-2, a, 3, x}\) e \(B = \set{-2, a, 1, 3, y}\) si ha che
        \[
            [A \centernot\subset B] \land [B \centernot\subset A].
        \]
    \end{enumerate}
\end{examples}


Siano \(A,B\) due insiemi. Dire che \(A \neq B\) equivale a dire, in simboli, che:
\begin{equation*}
    \begin{split}
        A \neq B &\iff \neg(A = B)
                  \iff \neg[(\forall x \in A) (x \in B) \land (\forall y \in B) (y \in A)]         \\
                 &\iff \neg[(\forall x \in A) (x \in B)] \land \neg[(\forall y \in B) (y \in A)]   \\
                 &\iff [(\exists x \in A)(x \notin B)] \lor [(\exists y \in B)(y \notin A)]
    \end{split}
\end{equation*}
Si deduce dunque che \(A \nsubseteq B \implies A \neq B\). Se \(A \subseteq B\) è contemplata la possibilità che \(A = B\), ad esempio \(\N \subseteq \N\) (\(\N\) è contenuto ed è uguale ad \(\N\)). In generale si ha che \(\forall A \text{ insieme } A \subseteq A\)


\begin{definition}[sottoinsieme proprio]\label{def:sottoinsieme_proprio}
    Si dice che \(A\) è un sottoinsieme proprio di \(B\) se \(A\) è contenuto in \(B\) ma è diverso da \(B\). Si utilizza il simbolo \(\subsetneq\).
\end{definition}
Dunque \(A \subsetneq B \iff \left[(A \subseteq B) \land (A \neq B)\right]\).

\begin{example}
    Tra gli insiemi numerici finora introdotti è presente la seguente relazione:
    \[
        \N \subsetneq \Z \subsetneq \Q \subsetneq \R.
    \]
\end{example}
Dire che \(A \subsetneq B\), equivale a dire, in simboli, che
\[
    A \subsetneq B \iff (A \subset B) \land (B \centernot\subset A)
                   \iff \left[\left((\forall x \in A) (x \in B)\right) \land \left((\exists x \in B) (x \notin A)\right)\right].
\]

\begin{definition}[insieme vuoto]\label{def:insieme_vuoto}
    L'unico insieme privo di elementi è l'insieme vuoto. Esso si indica col simbolo \(\emptyset\) e valgono le seguenti relazioni:
    \begin{itemize}
        \item \((\forall x)(x \notin \emptyset)\);
        \item \(\forall A \text{ insieme } \emptyset \subset A\)
    \end{itemize}
\end{definition}


\begin{proposition}[proprietà dell'inclusione]\label{prop:proprietà_inclusione}
    Siano \(A, B, C\) insiemi. Allora risulta:
    \begin{enumerate}
        \item \(A \subseteq A\);
        \item \([A \subseteq B] \land [B \subseteq A] \implies A = B\);
        \item \([A \subseteq B] \land [B \subseteq C] \implies A \subseteq C\).
    \end{enumerate}
\end{proposition}


\begin{definition}[insieme unione]\label{def:insieme_unione}
    Siano \(A, B\) due insiemi. Si dice insieme unione di \(A\) e \(B\) l'insieme
    \[
        A \cup B = \set{x \;\middle|\; [x \in A] \lor [x \in B]}.
    \]
\end{definition}
Si ha dunque che
\[
    x \in A \cup B \iff [x \in A] \lor [x \in B]
\]
L'unione di due insiemi può essere rappresentata come nella~\cref{fig:insieme_unione}

% !TEX root = ../Matematica discreta.tex

\begin{figure}[h]
\centering
    \begin{tikzpicture}

        \node (A)   at (-2.7, +1)  {\(A\)};
        \node (B)   at (+2.7, +1)  {\(B\)};
        \node (AuB) at (0.0, -1.2) {\(A \cup B\)};

        \draw
        [
            pattern = north west lines
        ]
        (-1, 0) ellipse (2cm and 1cm)
        (+1, 0) ellipse (2cm and 1cm);
    \end{tikzpicture}
\caption{Rappresentazione dell'insieme unione}\label{fig:insieme_unione}
\end{figure}


\begin{examples}
    Seguono vari esempi:
    \begin{enumerate}
        \item Siano \(A = \set{a, -1, b, -2}\) e \(B = \set{3, 2, 1}\), avremo che:
              \[
                  A \cup B = \set{a, -1, b, -2, 3, 2, 1};
              \]
        \item Siano \(X = \set{a, b, c, d}\) e \(Y = \set{b, e, f}\), si ha che:
              \[
                  X \cup Y = \set{a, b, c, d, e, f};
              \]
        \item Considerati \(\N\) e \(\Z\), \(\N \cup \Z = \Z\).
    \end{enumerate}
\end{examples}

\begin{proposition}[proprietà dell'unione]\label{prop:proprietà_unione}
    Siano \(A, B\) insiemi. Allora risulta:
    \begin{enumerate}
        \item \(A \cup \emptyset = A\);
        \item \([A \subset (A \cup B)] \land [B \subset (A \cup B)]\);
        \item \(A \cup A = A\);
        \item \(A \subset B \iff A \cup B = B\).
    \end{enumerate}
\end{proposition}


\begin{definition}[insieme intersezione]\label{def:insieme_intersezione}
    Siano \(A, B\) due insiemi. Si dice insieme intersezione di \(A\) e \(B\) l'insieme
    \[
        A \cap B = \set{x \;\middle|\; [x \in A] \land [x \in B]}.
    \]
\end{definition}
L'intersezione di due insiemi può essere rappresentata come nella~\cref{fig:insieme_intersezione}

% !TEX root = ../Matematica discreta.tex

\begin{figure}[h]
\centering
    \begin{tikzpicture}

        \node (A)   at (-2.7, +1)  {\(A\)};
        \node (B)   at (+2.7, +1)  {\(B\)};

        \draw (+1, 0) ellipse (2cm and 1cm);
        \draw (-1, 0) ellipse (2cm and 1cm);

        \clip (+1, 0) ellipse (2cm and 1cm);
        \fill[dashed, pattern=north west lines] (-1, 0) ellipse (2cm and 1cm);

        \node[preaction={fill, white}, rounded corners=3mm] (AuB) at (0.0, 0.0) {\(A \cap B\)};
    \end{tikzpicture}
\caption{Rappresentazione dell'insieme intersezione}\label{fig:insieme_intersezione}
\end{figure}

\begin{proposition}[proprietà dell'intersezione]\label{prop:proprietà_intersezione}
    Siano \(A, B\) due insiemi. Allora risulta:
    \begin{enumerate}
        \item \([A \cap B \subset A] \land [A \cap B \subset B]\);
        \item \(A \cap B = A \iff A \subset B\);
        \item \(A \cap \emptyset = \emptyset\);
        \item \(A \cap A = A\)
    \end{enumerate}
\end{proposition}

\begin{proposition}[proprietà di intersezione e unione]\label{prop:proprietà_unione_intersezione}
    Siano \(A, B, C\) insiemi. Allora:
    \begin{enumerate}
        \item Proprietà associative:
              \[
                    \left[(A \cup B) \cup C = A \cup (B \cup C)\right]
                    \land
                    \left[(A \cap B) \cap C = A \cap (B \cap C)\right];
              \]
        \item Proprietà commutative:
              \[
                    \left[A \cup B = B \cup A\right]
                    \land
                    \left[A \cap B = B \cap A\right];
              \]
        \item Proprietà distributive
              \[
                    \left[(A \cap B) \cup C = (A \cup C) \cap (B \cup C)\right]
                    \land
                    \left[(A \cup B) \cap C = (A \cup B) \cap (B \cup C)\right].
              \]
    \end{enumerate}
\end{proposition}


\begin{definition}[insieme complementare]\label{def:insieme_complementare}
    Siano \(A, B\) insiemi con \(B \subsetneq A\). Si dice insieme complementare di \(B\) rispetto ad \(A\) l'insieme
    \[
        \complement_A(B) = \set{x \in \isdiv{A}{x} \notin B}.
    \]
    Dunque
    \[
        x \in \complement_A(B) \iff [x \in A] \land [x \notin B].
    \]
\end{definition}
L'insieme complementare è rappresentato graficamente nella~\cref{fig:insieme_complementare}

% !TEX root = ../Matematica discreta.tex

\begin{figure}[h]
\centering
    \begin{tikzpicture}
        \node (A) at (-2.5, 1.5) {\(A\)};
        \node (B) at (1, 0)      {\(B\)};

        \draw
        [
            pattern = north west lines,
            even odd rule
        ]
        (0, 0) ellipse (3cm and 2cm)
        (1, 0) circle  (1cm);

        \node[preaction={fill, white}, rounded corners=3mm] (C-ab) at (-1, 0.5) {\(\complement_A(B)\)};
    \end{tikzpicture}
\caption{Insieme \(\complement_A(B)\), complementare di \(B\) rispetto ad \(A\)}\label{fig:insieme_complementare}
\end{figure}

\begin{examples}
    Seguono due esempi:
    \begin{enumerate}
        \item \(\complement_\Z(\N) = \set{x \in \\isdiv{Z}{n} \text{ è negativo}} = \set{\dotsc, -3, -2, -1}\)
        \item Presi \(A = \set{1, 2, 3, 4, 5, 6}\) e \(B = \set{1, 3, 5}\), si ha
        \[
            \complement_A(B) = \set{2, 4, 6}.
        \]
    \end{enumerate}
\end{examples}


\begin{proposition}[leggi di De Morgan, proprietà complementare]\label{prop:de_morgan_complementare}
    Siano \(A, B, C\) insiemi, con \(B,C \subsetneq A\). Risulta:
    \begin{enumerate}[label=\textnormal{\arabic*}.]
        \item \(\complement_A(A) = \emptyset\);
        \item \(\complement_A(\emptyset) = A\);
        \item \(B \cup \complement_A(B) = A\);
        \item \(\complement_A(B \cup C) = \complement_A(B) \cap \complement_A(C)\);
        \item \(\complement_A(B \cap C) = \complement_A(B) \cup \complement_A(C)\).
    \end{enumerate}
\end{proposition}
I punti 4. e 5. sono noti come \emph{leggi di De Morgan}. Dimostriamo la prima legge di De Morgan.

\begin{proof}
    Siano \(A, B, C\) insiemi, con \(B,C \subsetneq A\); per provare la prima legge di De Morgan è necessario provare che:
    \[
        \left[\complement_A(B \cup C) \subseteq \left(\complement_A(B) \cap \complement_A(C)\right)\right]
        \land
        \left[\left(\complement_A(B) \cap \complement_A(C)\right) \subseteq \complement_A(B \cup C)\right].
    \]
    Sia \(x \in A\). Dire che \(x \in \complement_A(B \cup C)\) equivale a dire che \((x \in A) \land (x \notin B \cup C)\), ovvero \((x \in A) \land \neg(x \in B \cup C)\). Ma questo è come scrivere \((x \in A) \land \neg(x \in B \lor x \in C)\), che negando la seconda parte diventa \((x \in A) \land (x \notin B \lor x \notin C)\). Ma allora questo è come dire che \((x \in A) \land (x \notin B) \lor (x \notin C)\)
    \begin{equation*}
        \begin{split}
            x \in \complement_A(B \cup C)%
                &\iff \left((x \in A) \land (x \notin B)\right) \lor \left((x \in A)(x \notin C)\right)   \\
                &\iff x \in \complement_A(B) \land x \in \complement_A(C)                                 \\
                &\iff x \in \complement_A(B) \cap \complement_A(C).                                       \\
        \end{split}
    \end{equation*}
    La legge risulta essere dimostrata.
    % La seconda legge risulta essere simile da dimostrare.
\end{proof}

\begin{definition}[insieme differenza]\label{def:insieme_differenza}
    Siano \(A, B\) insiemi. Si dice insieme differenza di \(A\) e \(B\) e si indica con i simboli \(A - B\) oppure \(A \setminus B\) l'insieme:
    \[
        A \setminus B = \set{x \in \isdiv{A}{x} \notin B}.
    \]
\end{definition}

\begin{remark}
    Siano \(A, B\) insiemi. Se \(B \subseteq A\) allora \(A \setminus B = \complement_A(B)\)
    % !TEX root = ../Matematica discreta.tex

\begin{figure}[h]
\centering
    \begin{tikzpicture}
        \node (A)   at (-2.7, +1)  {\(A\)};
        \node (B)   at (+2.7, +1)  {\(B\)};

        \begin{scope}
            \clip (-1, 0) ellipse (2cm and 1cm);
            \draw[
                pattern = north west lines,
                even odd rule,
            ] (+1, 0) ellipse (2cm and 1cm)
              (-1, 0) ellipse (2cm and 1cm);
        \end{scope}

        \draw (+1, 0) ellipse (2cm and 1cm);
        \draw (-1, 0) ellipse (2cm and 1cm);

        \node[preaction={fill, white}, rounded corners=3mm] (A-B) at (-1.8, 0) {\(A \setminus B\)};
    \end{tikzpicture}
\caption{Insieme differenza tra due insiemi \(A, B\) nel caso in cui \(B \subseteq A\).}\label{fig:insieme_differenza_oss}
\end{figure}
    Dunque \(A \setminus B = \complement_A(A \cap B)\).
\end{remark}

\begin{examples}
    Seguono due esempi:
    \begin{enumerate}
        \item Siano \(A = \set{x, y, z}\) e \(B = \set{a, b, x}\). Si ha che:
            \[
                A \setminus B = \set{y, z}, \qquad
                B \setminus A = \set{a, b}.
            \]
        \item L'insieme dei numeri irrazionali può essere indicato con \(\R \setminus \Q\).
    \end{enumerate}
\end{examples}

\begin{definition}[insieme delle parti]\label{def:insieme_parti}
    Sia \(A\) un insieme. L'insieme di tutti i sottoinsiemi di \(A\) si dice insieme delle parti di \(A\). Si indica col simbolo
    \[
        \powerset(A)
    \]
\end{definition}

\begin{examples}
    Seguono degli esempi:
    \begin{enumerate}
        \item Sia \(A = \set{a, b, c}\), si ha
        \[
            \powerset(A) = \set{\emptyset, \set{a}, \set{b}, \set{c} \set{a, b}, \set{b, c}, \set{a, c}, \set{a, b, c}};
        \]
        \item \(\powerset(\emptyset) = \set{\emptyset}\);
        \item Sia \(B = \set{1}\), si ha \(\powerset(B) = \set{\emptyset, \set{1}}\).
    \end{enumerate}
\end{examples}


\begin{remark}
    Sia \(A\) un insieme. Si ha che \([\emptyset \in \powerset(A)] \land [A \in \powerset(A)]\)
\end{remark}


\begin{definition}[insiemi disgiunti]\label{def:insiemi:disgiunti}
    Siano \(A, B\) insiemi. Si dice che \(A\) e \(B\) sono disgiunti quando \(A \cap B = \emptyset\)
\end{definition}


\begin{examples}
    Seguono due esempi:
    \begin{enumerate}
        \item Sia \(P\) l'insieme dei numeri interi pari e sia \(D\) l'insieme dei numeri interi dispari. Si ha
            \[
                P \cap D = \emptyset
            \]
            e dunque \(P\) e \(D\) sono disgiunti. Inoltre \(P \cup D = \Z\).
        \item Siano \(A = \set{1, 2, 3, 4}\) e \(B = \set{0, 6, 9, 8}\). Si ha che
            \[
                A \cap B = \emptyset
            \]
            e quindi \(A\) e \(B\) sono disgiunti.
    \end{enumerate}
\end{examples}

\begin{remark}
    Siano \(a, b\) due oggetti. Si ha che \(\set{a, b} = \set{b, a}\). È quindi irrilevante l'ordine in cui si scrivono gli elementi. Per questo \(\set{a, b}\) si chiama coppia non ordinata.

    Siano \(A\) un insieme e \(a, b \in A\). Si indica col simbolo \((a, b)\) la coppia ordinata la cui prima coordinata è \(a\) e la seconda coordinata è \(b\). Si ha che \((a, b) \neq (b, a)\).
\end{remark}
Naturalmente si possono considerare le coppie ordinate di elementi di insiemi diversi: se \(X, Y\) sono due insiemi, con \(x \in X\) e \(y \in Y\) si ha che \((x, y) \neq (y, x)\).


\begin{definition}[prodotto cartesiano]\label{def:profotto_cartesiano}
    Siano \(A, B\) due insiemi. L'insieme i cui elementi sono tutte le possibili coppie aventi prima coordinata un elemento di \(A\) e seconda coordinata un elemento di \(B\) si dice prodotto cartesiano di \(A\) e \(B\). Questo insieme si indica col simbolo \(A \times B\). Quindi
    \[
        A \times B = \set{(a, b) \middle| [a \in A] \land [b \in B]}
    \]
\end{definition}

\begin{remark}
    Siano \(A, B\) due insiemi con \(A \neq B\). Si ha che
    \[
        A \times B \neq B \times A.
    \]
\end{remark}

\begin{example}
    Un esempio di prodotto cartesiano è \(\R \times \R\)
    % !TEX root = ../Matematica discreta.tex

\begin{figure}[h]
\centering
    \begin{tikzpicture}
        % draw the axis
        \draw[-latex] (-1, 0) -- (3, 0);   % x axis
        \draw[-latex] (0, -1) -- (0, 3);   % y axis

        % where 1 and origin is on axis
        \node (1-x) [below]    at (1, 0) {\(1\)};
        \node (1-y) [left]     at (0, 1) {\(1\)};
        \node (0) [below left] at (0, 0) {\(0\)};

        % x0 and y0 coords
        \node (x0) [below] at (2, 0) {\(x_0\)};
        \node (y0) [left]  at (0, 2) {\(y_0\)};

        % the meeting point
        \coordinate (x0-y0) at (2, 2);
        \node at (x0-y0) [above right] {\((x_0, y_0)\)};

        % draw lines and dot
        \draw[dashed, thin] (x0) -- (x0-y0) -- (y0);
        \filldraw[black] (x0-y0) circle (1pt);
    \end{tikzpicture}
\caption{Generico punto di coordinate \((x_0, y_0)\) all'interno del piano cartesiano \(\R \times \R\)}\label{fig:punto_piano_cartesiano}
\end{figure}
    Il piano cartesiano può essere rappresentato con \(\R \times \R\).
\end{example}

\begin{example}
    Siano \(A = \set{a, b, c}\) e \(B = \set{1, 2, 3, 4}\). Si ha che
    \begin{alignat*}{1}
            A \times B = \{
                &(a, 1), (a, 2), (a, 3), (a, 4), \\
                &(b, 1), (b, 2), (b, 3), (b, 4), \\
                &(c, 1), (c, 2), (c, 3), (c, 4)%
            \},
    \end{alignat*}
    mentre
    \begin{alignat*}{1}
        B \times A = \{
            &(1, a), (1, b), (1, c), \\
            &(2, a), (2, b), (2, c), \\
            &(3, a), (3, b), (3, c), \\
            &(4, a), (4, b), (4, c)%
        \}.
    \end{alignat*}
    Dunque \(A \times B \neq B \times A\) e in questo caso \(A \times B \cap B \times A = \emptyset\).
\end{example}


\section{Relazioni}\label{sec:relazioni}

\begin{definition}[relazione]\label{def:relazione}
    Siano \(A, B\) due insiemi. Una relazione tra gli elementi di \(A\) e quelli di \(B\) è un sottoinsieme del prodotto cartesiano \(A \times B\).
\end{definition}
In altri termini una relazione tra gli elementi di \(A\) e gli elementi di \(B\) è un insieme di coppie ordinate la cui prima coordinata è un elemento di \(A\) mentre la seconda coordinata è un elemento di \(B\).

\begin{examples}
    Seguono degli esempi:
    \begin{enumerate}
        \item Siano \(A = \set{a, b, c}\) e \(B = \set{1, 2, 3, 4}\). Si ha
            \[
                \rel[0]{A}{B} = \set{(a, 1)} \subseteq A \times B
            \]
            dove \(\rel[0]{A}{B}\) è una relazione tra gli elementi di \(A\) e gli elementi di \(B\). Anche
            \[
                \rel[1]{A}{B} = \set{(b, 1), (b, 2), (c, 3), (c, 4)}
            \]
            è una relazione e \(\rel[1]{A}{B} \subseteq A \times B\);
        \item una relazione su \(\R \times \R\) può essere la seguente:
            \[
                \rel[2]{\R} = \set{(\sqrt{2}, 3), (5, -2), (-\sqrt{3}, -\sqrt{3})}.
            \]
            Di nuovo, \(\rel[2]{\R} \subseteq \R \times \R\);
        \item una relazione di \(\Z \times \Z\) è la seguente:
            \[
                \rel[3]{\Z} = \set{(a, b) \in \Z \times \Z : b = 2a}
            \]
            e ancora \(\rel[3]{\Z} \subseteq \Z \times \Z\) e si ha
            \[
                \rel[3]{\Z} = \set{\dotsc, (0, 0), (1, 2), (-1, -2), (3, 6), \dotsc};
            \]
        \item una relazione su \(\N \times \Z\) è:
            \[
                \rel[4]{\N \times \Z} = \set{(n, m) \in \N \times \Z : m = -n} \subseteq \Z \times \Z.
            \]
            \(\rel[4]{\N}{\Z}\) è una relazione tra gli elementi di \(\N\) e quelli di \(\Z\);
        \item una relazione su \(\Z \times \Z\) è:
            \[
                \rel[5]{\Z} = \set{(n, m) \in \Z \times \Z : m = -n} \subseteq \Z \times \Z.
            \]
            Notiamo che \(\rel[4]{\N}{\Z} \neq \rel[5]{\Z}\), visto che il prodotto cartesiano prende in considerazione due insiemi differenti.
    \end{enumerate}
\end{examples}

\begin{example}
    Sia \(A\) un insieme. Si dice diagonale di \(A\) l'insieme:
    \[
        \setdiag{A} = \set{(a, b) \in A \times \isdiv{A}{a} = b}
    \]
\end{example}

\begin{remark}
    Sia \(A\) un insieme. Si ha
    \[
        A \times \emptyset = \emptyset \times A = \emptyset.
    \]
    Per questo si considerano soltanto relazioni tra gli elementi di due insiemi non vuoti.
\end{remark}


\begin{definition}[relazione riflessiva]\label{def:relazione_riflessiva}
    Siano \(A\) un insieme non vuoto e \(\rel{A}\) una relazione tra gli elementi di \(A\) (o semplicemente su \(A\)). Si dice che \(\rel{A}\) è riflessiva se
    \[
        \forall a \in A \; (a, a) \in \rel{A}
    \]
\end{definition}

\begin{example}
    La bisettrice del \(1^\text{\!o}\) e \(3^\text{o}\) quadrante rappresenta la diagonale \(\setdiag{\R}\). La diagonale è riflessiva.
    % !TEX root = ../Matematica discreta.tex

\begin{figure}[h]
\centering
    \begin{tikzpicture}
        \begin{axis}[
                xmin = -2.5,   xmax = 2.5,
                ymin = -2.5,   ymax = 2.5,
                x post scale=0.5,
                y post scale=0.5,
                axis lines=center,
                axis on top=true,
                domain=-2:2,
                xticklabels={,,},
                yticklabels={,,},
                ticks=none,
            ]

            \addplot[color = black] {(\x)};
        \end{axis}
    \end{tikzpicture}
\caption{Bisettrice del \(1^\text{\!o}\) e \(3^\text{o}\) quadrante}\label{fig:bisettrice_1-3_quadr}
\end{figure}
\end{example}

\begin{remark}
    Siano \(A\) un insieme non vuoto e \(\rel{A}\) una relazione su \(A^2\). \(\rel{A}\) è riflessiva se e solo se \(\setdiag{A} \subseteq \rel{A}\).
\end{remark}

\begin{remark}
    Siano \(A\) un insieme non vuoto e \(\rel{A}\) una relazione su \(A^2\). Allora, perché \(\rel{A}\) non sia riflessiva basta che \(\exists x \in A \tc (x, x) \notin \R\).
\end{remark}


\begin{example}
    Sia \(X\) l'insieme dei residenti a Bari. La relazione
    \[
        \rel{X} = \set{(x, y) \in X \times \isdiv{X}{x} \text{ ha la stessa madre di } y}
    \]
    risulta essere simmetrica e riflessiva.
\end{example}

\begin{definition}[relazione simmetrica]\label{def:relazione_simmetrica}
    Siano \(A\) un insieme non vuoto e \(\rel{A}\) una relazione su \(A^2\). Si dice che \(\rel{A}\) è simmetrica se
    \[
        \left(\forall x,y \in A\right)\left((x, y) \in \rel{A} \implies (y, x) \in \rel{A} \right).
    \]
\end{definition}

\begin{definition}[relazione antisimmetrica]\label{def:relazione_antisimmetrica}
    Siano \(A\) un insieme non vuoto e \(\rel{A}\) una relazione su \(A^2\). Si dice che \(\rel{A}\) è antisimmetrica se
    \[
        \left(\forall a,b \in A\right) \left((a, b) \in \rel{A} \land (b, a) \in \rel{A}\right) \implies a = b.
    \]
    Equivalentemente
    \[
        (\forall a,b \in A, \; a \neq b) \left((a, b) \in \rel{A} \implies (b, a) \notin \rel{A}\right).
    \]
\end{definition}

\begin{remark}
    Una relazione simmetrica \textbf{non} può essere antisimmetrica e una relazione antisimmetrica \textbf{non} può essere simmetrica, a esclusione della diagonale \(\Delta\).
\end{remark}

\begin{example}
    {\Large wtf}\\
    Siano \(a,b \in \R\) con \(a \leq b \land b \leq a \implies a = b\). Sia \(\rel{\R}\) una relazione e sia \((a, b) \in \rel{\R}\)
\end{example}


\begin{definition}[relazione transitiva]\label{def:relazione_transitiva}
    Siano \(A\) un insieme non vuoto e \(\rel{A}\) una relazione su \(A^2\). Si dice che \(\rel{A}\) è transitiva se
    \[
        (\forall a,b,c \in A) \left((a, b) \in \rel{A} \land (b, c) \in \rel{A}\right) \implies (a, c) \in \rel{A}
    \]
\end{definition}


\begin{examples}
    Sia \(A = \set{a, b, c, d}\). Seguono una serie di relazioni su \(A^2\):
    \begin{enumerate}
        \item la relazione:
            \[
                \rel[1]{A} = \set{(a, a), (a, b), (b, b), (c, c), (d, d)}
            \]
            risulta essere riflessiva ma non simmetrica, perché \((a, b) \in \rel[1]{A}\) ma \((b, a) \notin \rel[1]{A}\), è antisimmetrica ed è transitiva;
%
        \item la relazione:
            \[
                \rel[2]{A} = \set{(a, a), (a, b), (b, a)}
            \]
            non è riflessiva perché \((b, b), (c, c), (d, d) \notin \rel[2]{A}\) e non è transitiva ma è simmetrica perché \((a, b) \land (b, a) \in \rel[2]{A}\);
%
        \item la relazione:
            \[
                \rel[3]{A} = \set{(a, b), (b, c), (a, c)}
            \]
            non è né riflessiva, né simmetrica ma è sia antisimmetrica che transitiva;
%
        \item la relazione:
            \[
                \rel[4]{A} = \set{(a, a), (b, b), (c, c), (d, d), (a, b), (b, a), (a, c), (c, a), (b, c), (c, b)}
            \]
            è sia riflessiva, sia simmetrica e sia transitiva;
%
        \item la relazione:
            \[
                \rel[5]{A} = \set{(a, a), (b, b), (c, c), (d, d), (a, b), (b, c), (a, c)}
            \]
            è riflessiva, antisimmetrica e transitiva, ma non è simmetrica;
%
        \item la relazione:
            \[
                \rel[6]{A} = \set{(a, a), (b, b), (d, d), (a, b), (b, c)}
            \]
            non è né riflessiva, perché \((c, c) \notin \rel[6]{A}\), né simmetrica, né transitiva, ma è antisimmetrica;
%
        \item la relazione:
            \[
                \rel[7]{A} = \set{(a, b), (b, c), (a, c), (b, a), (c, b), (c, a)}
            \]
            non è riflessiva e nemmeno transitiva, però è simmetrica;
%
        \item la relazione:
            \[
                \rel[8]{A} = \set{(a, b), (b, c), (a, c), (b, a), (c, b)}
            \]
            non è né simmetrica, perché \((a, c) \in \rel[8]{A}\) ma \((c, a) \notin \rel[8]{A}\), né riflessiva, né antisimmetrica e nemmeno transitiva;
%
        \item la relazione:
            \[
                \rel[9]{A} = \set{(a, b), (b, c), (a, c), (b, a)}
            \]
            non è né simmetrica, né riflessiva, né transitiva e nemmeno antisimmetrica.
    \end{enumerate}
\end{examples}

\begin{example}
    Sia \(\Pi\) l'insieme delle rette di un piano fissato. Sia \(\rel{\Pi}\) una relazione su \(\Pi^2\):
    \[
        \rel{\Pi} = \set{(r, s) \in \Pi \times \Pi \mid r \text{ ha la stessa direzione di } s}
    \]
    % TODO: (2) - @to-complete [example] - \rel{\Pi} - l. 18/10/'21
\end{example}


\subsection{Relazioni d'ordine}\label{subsec:relazioni_d'ordine}

\begin{definition}[relazione d'ordine]\label{def:relazione_d'ordine}
    Siano \(A\) un insieme non vuoto e \(\rel{A}\) una relazione su \(A^2\). Si dice che \(\rel{A}\) è una relazione d'ordine se è riflessiva, antisimmetrica e transitiva.
\end{definition}

\begin{remark}
    Sia \(\rel{A}\) una relazione d'ordine sull'insieme \(A\), essa viene denotata col simbolo \(\leq\). Dunque
    \[
        \forall a,b \in A \quad (a, b) \in \rel{A} \text{ si scrive } a \leq b.
    \]
\end{remark}

Ad esempio, le relazioni \(\rel[1]{A}\) e \(\rel[5]{A}\), enunciate nella serie di esempi che segue la~\cref{def:relazione_transitiva}, sono due relazioni d'ordine.
Esplicitiamo la \(\rel[1]{A}\) tramite l'uso del simbolo della relazione d'ordine \(\leq\):
\[
    a \leq a, \quad
    b \leq b, \quad
    c \leq c, \quad
    d \leq d, \quad
    a \leq b.
\]
Esplicitiamo ora la \(\rel[5]{A}\):
\[
    a \leq a, \quad
    b \leq b, \quad
    c \leq c, \quad
    d \leq d, \quad
    a \leq b, \quad
    b \leq c, \quad
    a \leq c.
\]

\begin{example}
    Definiamo la relazione d'ordine naturale su  \(\Z\)
    \[
        (\forall n,m \in \Z) \left(n \leq m \iff \exists h \in \N : m = h + n \right).
    \]
    Ad esempio \(-7 \leq -5\) perché \(-5 = 2 - 7\), dove \(h = 2\).
    Dunque \(\leq\) è una relazione d'ordine su \(\Z\).
\end{example}

\begin{proof}
    dimostriamo che \(\leq\) è una relazione d'ordine su \(\Z\). Bisogna dimostrare che \(\leq\) è sia riflessiva che antisimmetrica che transitiva. Dimostriamo che \(\leq\) è riflessiva, ovvero che
    \[
        (\forall n \in \N) \left(n \leq n\right).
    \]
    Basta prendere \(h = 0\), infatti:
    \[
        (\forall n \in \N) \left(\exists h = 0 \in \N : n = 0 + n\right).
    \]
    Dimostriamo ora che \(\leq\) è antisimmetrica, ovvero che
    \[
        (\forall n,m \in \Z) \left((n \leq m \land m \leq n) \implies (n = m)\right).
    \]
    Siano dunque \(n,m \in \Z\) con \((n \leq m \land m \leq n)\), questo vuol dire che
    \[
        \left(\exists h \in \N : m = h + n\right)
        \land
        \left(\exists h \in \N : n = h + m\right).
    \]
    Si ha dunque
    \[
        m = h + n
          = h + (k + m)
          = (h + k) + m,
    \]
    ma allora \(m = (h + k) + m\), dunque \(h + k = 0\), però \(h,k \in \N\) allora necessariamente \(h,k = 0\)\footnote{%
        Proprietà dell'annullamento della somma in \(\N\):
        \[
            (\forall r,s \in \N) \left(r + s = 0 \implies r = 0 \land s = 0\right)
        \]
    }, pertanto \(m = 0 + n\), ovvero \(m = n\).
    Dimostriamo che \(\leq\) è antisimmetrica. Siano \(n,m,p \in \Z\), con \(n \leq m \land m \leq p\); è necessario verificare che \(n \leq p\). Dire che \(n \leq m \land m \leq p\) implica
    \[
        \left(\exists h \in \N : m = h + n\right)
        \land
        \left(\exists k \in \N : p = k + m\right),
    \]
    allora
    \[
        p = k + m
          = k + (h + n)
          = (k + h) + n.
    \]
    Sia \(t = k + h\), allora
    \[
        \exists t \in \N : p = t + n \implies n \leq p.
    \]
    \(\leq\) è effettivamente una relazione d'ordine su \(\Z\).
\end{proof}

\begin{examples}
    Sia \(A\) un insieme. La relazione di inclusione \(\subseteq\) è una relazione d'ordine su \(\powerset(A)\)
    \begin{enumerate}
        \item \(\forall X     \in \powerset(A)\;  X \subseteq X \), \(\subseteq\) è riflessiva;
        \item \(\forall X,Y   \in \powerset(A)\; [X \subseteq Y \land Y \subseteq X] \implies X = Y \), \(\subseteq\) è antisimmetrica;
        \item \(\forall X,Y,Z \in \powerset(A)\; [X \subseteq Y \land Y \subseteq Z] \implies X \subseteq Z\), \(\subseteq\) è transitiva;
    \end{enumerate}
\end{examples}



\begin{definition}[divisibilità]\label{def:divisibilità}
    Siano \(a,b \in \Z\) con \(b \neq 0\). Si dice che \(b\) divide \(a\) oppure che \(b\) è divisore di \(a\) oppure, allo stesso modo, che \(a\) è multiplo di \(b\) oppure ancora che \(a\) moltiplica \(b\) e si scrive:
    \[
        \isdiv{a}{b}
    \]
    se e solo se \(\exists h \in \Z : a = h \cdot b\).
\end{definition}

\begin{example}
    Si ha ad esempio
    \[
        \isdiv{3}{12}, \quad
        \isdiv{-2}{8}, \quad
        \isdiv{5}{0}.
    \]
    questo perché \(\exists h = 4 \in \Z : 12 = 3h\), \(\exists k = -4 \in \Z : 8 = -2k\) e \(\exists t = 0 \in \Z : 5 = 5t\).
\end{example}

\begin{remark}
    \(\forall b \in \Z\), con \(b \neq 0\) si ha \(\isdiv{b}{0}\). Infatti \(\exists h = 0 \in \Z : 0 = hb\)
\end{remark}
\begin{remark}
    \(\forall a \in \Z_*\), si ha
    \[
        \isdiv{a}{a}, \quad
        \isdiv{a}{{-}a}, \quad
        \isdiv{{-}a}{a}, \quad
        \isdiv{{-}a}{{-}a}.
    \]
\end{remark}
\begin{remark}
    \(\forall a \in \Z\), si ha
    \[
        \isdiv{1}{a} \land \isdiv{-1}{a}.
    \]
\end{remark}


\begin{proposition}[proprietà divisibilità]\label{prop:proprietà_divisibilità}
    Siano \(a,b,c \in \Z\) con \(a \neq 0\), risulta:
    \begin{enumerate}
        \item \(\isdiv{a}{b} \implies \left(\isdiv{a}{{-}b} \land \isdiv{{-}a}{b} \land \isdiv{{-}a}{{-}b}\right)\);
        \item \(\left(b \neq 0 \land \isdiv{a}{b} \land \isdiv{b}{a}\right) \implies a = \pm b\);
        \item \((\isdiv{a}{b} \land \isdiv{b}{c}) \implies \isdiv{a}{c}\);
        \item \((\isdiv{a}{b} \land \isdiv{a}{c}) \implies \isdiv{a}{b \pm c}\);
        \item \(\isdiv{a}{b} \implies \isdiv{a}{bc}\).
    \end{enumerate}
\end{proposition}

\begin{proof}
    dimostriamo la 1. e la 2.
    % TODO: (0) - @to-complete [proof] - propr. a|b - l. 18/10/'21
\end{proof}

% TODO: (4) - @to-complete [text] - \mid è una rel ordinata su \N - l. 18/10/'21


\begin{definition}[insieme totalmente ordinato]\label{def:insieme_totalmente_ordinato}
    Sia \((A, \leq)\) un insieme ordinato. Si dice che \((A, \leq)\) è totalmente ordinato se
    \[
        (\forall x,y \in A) (x \leq y \lor y \land x).
    \]
    Si dirà che è parzialmente ordinato se
    \[
        \exists x,y \in A : x \nleq y \land y \nleq x.
    \]
\end{definition}

Ad esempio, \((\Z, \leq)\) è totalmente ordinato e \((\powerset(A), \subseteq)\) è parzialmente ordinato.

\begin{example}
    Sia \(A = \set{1, 2, 3, 4}\). Si ha che:
    \begin{equation*}
    \begin{split}
        \powerset(A) = \{\emptyset,
            \setsc{1, 2, 3, 4, {1, 2}, {1, 3}, {1, 4}, {2, 3}, {2, 4}, {3, 4}}, \\
            \setsc{{1, 2, 3}, {1, 2, 4}, {1, 3, 4}, {2, 3, 4}, {1, 2, 3, 4}}
        \}.
    \end{split}
    \end{equation*}
    Prendiamo, ad esempio, i due sottoinsiemi di \(\powerset(A)\), \(\set{1, 2}\) e \(\set{1, 3, 4}\), notiamo che
    \[
        \set{1, 2} \nsubseteq \set{1, 3, 4}
        \qquad
        \set{1, 3, 4} \nsubseteq \set{1, 2}.
    \]
\end{example}

\begin{example}
    L'insieme \((\N_*, \mid)\) è parzialmente ordinato, infatti
    \[
        \isndiv{5}{16} \land \isndiv{16}{5}
    \]
\end{example}

\begin{example}
    Sia \(D_{20} = \set{1, 2, 4, 5, 10, 20}\). L'insieme \((D_{20}, \mid)\) è un insieme ordinato:
    \[
        \isdiv{1}{2}, \quad
        \isdiv{1}{4}, \quad
        \dotsc,       \quad
        \isdiv{2}{4}, \quad
        \isdiv{2}{10},
    \]
    ma non è totalmente ordinato, infatti:
    \[
        \isndiv{4}{10} \land \isndiv{10}{4}.
    \]
\end{example}

In generale, dato \((A, \leq)\) insieme ordinato e \(B \subseteq A\), con \(B \neq \emptyset\), si può considerare la relazione d'ordine indotta \(\leq_B\):
\[
    \forall a,b \in B \quad a \leq_B b \iff a \leq b.
\]
Dunque anche \((B, \leq_B)\) è un insieme ordinato.


\begin{definition}[minimo e massimo]\label{def:minimo_massimo}
    Sia \((A, \leq)\) un insieme ordinato e sia \(B \subseteq A\). Sia \(x_0 \in B\), esso si dice
    \begin{itemize}
        \item \emph{minimo} di \(B\) se:
            \[
                \forall x \in B \quad x_0 \leq x;
            \]
        \item \emph{massimo} di \(B\) se:
            \[
                \forall x \in B \quad x \leq x_0.
            \]
    \end{itemize}
    Se esiste il minimo di \(B\) (anche detto il più piccolo elemento di \(B\)) si indica con \(\min{(B)}\); se esiste il massimo di \(B\) (anche detto il più grande elemento di \(B\)) si indica con \(\max{(B)}\).
\end{definition}


\begin{proposition}[unicità massimo e minimo]\label{prop:unicità_massimo_minimo}
    Sia \((A, \leq)\) un insieme ordinato e sia \(B \subseteq A\). Se esiste un minimo (o un massimo) di \(B\), esso è unico.
\end{proposition}

\begin{proof}
    Siano \(x_0, y_0 \in B\) due minimi di \(B\). Si ha
    \[
        \forall x \in B \;\; x_0 \leq x
        \land
        \forall x \in B \;\; y_0 \leq x.
    \]
    In particolare
    \[
        (x_0 \leq y_0 \land y_0 \leq x_0) \implies x_0 = y_0.
    \]
    Come volevasi dimostrare.
\end{proof}\\

\begin{examples}
    Seguono degli esempi di insiemi ordinati o meno che abbiano massimo, minimo o nessuno dei due.
    \begin{enumerate}
        \item \((\N, \leq)\) ammette minimo, che è \(0\) ma non ammette massimo;
        \item \((\Z, \leq)\) non ammette né minimo, né massimo;
        \item \((\N_*, \leq)\) ammette minimo che è \(1\) ma non ammette massimo;
        \item \(D_{20}\) ammette minimo che è \(1\) e massimo che è \(20\);
        \item \((\powerset(A), \subseteq)\) ammette minimo che è \(\emptyset\) e massimo che è \(A\);
        \item sia \(A = \set{-2, 0, 1, 2, 3}\). \(A\) ha minimo che è \(-2\) e massimo che è \(3\).
    \end{enumerate}
\end{examples}


\subsection{Relazioni d'equivalenza}\label{sec:Relazioni_d'equivalenza}

\begin{definition}[relazione d'equivalenza]\label{def:relazione_d'equivalenza}
    Sia \(A\) un insieme non vuoto. Una relazione \(\rel{}\) su \(A\) si dice di equivalenza se è riflessiva, simmetrica e transitiva.
\end{definition}

\begin{example}
    Sia \(\Pi\) l'insieme delle rette del piano. La relazione
    \[
        \rel[1]{\Pi} = \set{(r, s) \in \Pi \times \Pi \mid r \text{ ha la stessa direzione di } s},
    \]
    è una relazione di equivalenza. La relazione
    \[
        \rel[2]{\Pi} = \set{(r, s) \in \Pi \times \Pi \mid r \text{ è perpendicolare ad } s},
    \]
    non è di equivalenza e nemmeno d'ordine.
\end{example}

\begin{example}
    Sia \(A\) l'insieme dei residenti a Bari e sia \(\rel[1]{A}\) una relazione su \(A^2\):
    \[
        (\forall x,y \in A) (x, y) \in \R \iff x \text{ ha la stessa madre di } y.
    \]
    \(\rel[1]{A}\) è riflessiva, simmetrica e transitiva e quindi è di equivalenza.
\end{example}

\begin{example}
    Sia \(A\) un insieme non vuoto. La diagonale
    \[
        \setdiag{A} = \set{(x, y) \in A \times A \mid x = y}
    \]
    è una relazione sia d'equivalenza che d'ordine (è l'unica che verifica entrambe le definizioni).
\end{example}


\begin{definition}[classe d'equivalenza]\label{def:classe_d'equivalenza}
    Siano \(A\) un insieme non vuoto e \(\rel{A}\) una relazione di equivalenza su \(A\). Per ogni elemento \(a\) di \(A\) si dice classe di equivalenza di \(a\) rispetto a \(\rel{A}\) il sottoinsieme di \(A\):
    \[
        \eqvclass{a}{A^2} = \set{b \in A \mid (a, b) \in \rel{A}}.
    \]
\end{definition}

\begin{examples}
    Seguono una serie di esempi
    \begin{enumerate}
        \item Sia \(A = \set{a, b, c}\) e sia
            \[
                \rel{A} = \set{(a, a), (b, b), (c, c), (a, b), (b, a)}.
            \]
            Si ha
            \[
                \eqvclass{a}{A^2} = \set{a, b} =\eqvclass{b}{A^2},
                \quad
                \eqvclass{c}{A^2} = \set{c}.
            \]
%
        \item La relazione:
            \[
                \rel[\prime]{\Z} = \set{(x, y) \in \Z \times \Z \mid x^2 = y^2};
            \]
            risulta essere:
            \begin{itemize}
                \item riflessiva perché:
                    \[
                        \forall x \in \Z \quad x^2 = x^2 \implies (x, x) \in \rel[\prime]{\Z};
                    \]
    %
                \item simmetrica perché:
                    \begin{alignat*}{1}
                        \forall x,y \in \Z \quad (x, y) \in \rel[\prime]{\Z}
                            &\implies x^2 = y^2   \\
                            &\implies y^2 = x^2   \\
                            &\implies (y, x) \in \rel[\prime]{\Z};
                    \end{alignat*}
    %
                \item transitiva perché:
                    \begin{alignat*}{1}
                        \forall x,y,z \in \Z \quad (x, y) \in \rel[\prime]{\Z} \land (y, z) \in \rel[\prime]{\Z}
                            &\implies x^2 = y^2 \land y^2 = z^2   \\
                            &\implies x^2 = z^2                   \\
                            &\implies (x, z) \in \rel[\prime]{\Z}.
                    \end{alignat*}
    %
            \end{itemize}
            Si ha, per esempio:
            \[
               \eqvclass{0}{\Z^2}[\prime] = \set{0},                                     \quad
               \eqvclass{1}{\Z^2}[\prime] = \set{1, -1} =\eqvclass{-1}{\Z^2}[\prime], \quad
               \eqvclass{2}{\Z^2}[\prime] = \set{2, -2} =\eqvclass{-2}{\Z^2}[\prime].
            \]
            Si evince che per ogni \(x \in \Z\), per \(x \neq 0\),
            \[
               \eqvclass{x}{\Z^2}[\prime] = \set{-x, x}.
            \]
%
        \item La relazione:
            \[
                \rel[\second]{A} = \set{(n, m) \in \Z_* \times \Z_* \mid n \cdot m > 0}
            \]
            risulta essere di equivalenza, perché è:
            \begin{itemize}
                \item riflessiva, infatti:
                    \[
                        \forall n \in \Z_* \quad n \cdot n = n^2 > 0 \implies (n, n) \in \rel[\second]{\Z_*};
                    \]
    %
                \item simmetrica, infatti:
                    \begin{alignat*}{1}
                        \forall n,m \in \Z_* \quad (n, m) \in \rel[\second]{\Z}
                            &\implies n \cdot m > 0 \\
                            &\implies m \cdot n > 0 \\
                            &\implies (m, n) \in \rel[\second]{\Z}
                    \end{alignat*}
    %
                \item transitiva, infatti:
                    \begin{alignat*}{1}
                        \forall n,m,p \in \Z_* \quad (n, m) \in \rel[\second]{\Z} \land (m, p) \in \rel[\second]{\Z}
                            &\implies n \cdot m \cdot m \cdot p > 0 \\
                            &\implies np \cdot m^2 > 0 \\
                            &\implies np > 0 \\
                            &\implies (n, p) \in \rel[\second]{\Z}.
                    \end{alignat*}
    %
            \end{itemize}
        Si ha, per esempio:
        \[
           \eqvclass{1}{\Z^2}[\second] = \set{1, 2, 3, \dotsc, n}, \quad
           \eqvclass{-1}{\Z^2}[\second] = \set{-1, -2, -3, \dotsc, -n}.
        \]
    \end{enumerate}
\end{examples}


\begin{proposition}[proprietà classi d'equivalenza]\label{prop:proprietà_classi_di_equivalenza}
    Siano \(A\) un insieme non vuoto e \(\rel{A}\) una relazione di equivalenza su \(A\). Allora risulta:
    \begin{enumerate}
        \item \(\forall a \in A \emspace{1.85} \eqvclass{a}{A^2} \neq \emptyset\);
        \item \(\forall a,b \in A \quad \eqvclass{a}{A^2} = \eqvclass{b}{A^2} \iff (a, b) \in \rel{A}\);
        \item \(\forall a,b \in A \quad \eqvclass{a}{A^2} \cap \eqvclass{b}{A^2} = \emptyset \iff (a, b) \notin \rel{A}\).
    \end{enumerate}
\end{proposition}

\begin{proof}
    Dimostriamo i \(3\) punti precedenti:
    \begin{enumerate}
        \item \(\forall a \in A \quad \eqvclass{a}{A^2} \neq 0\) perché \(a \in \eqvclass{a}{A^2}\) in quanto \((a, a) \in \rel{A}\);
        \item dimostriamo la doppia implicazione:
            \begin{lhs}
                È necessario provare che \(\eqvclass{a}{A^2} = \eqvclass{b}{A^2}\), ovvero
                \[
                    \eqvclass{a}{A^2} \subseteq \eqvclass{b}{A^2}
                    \land
                    \eqvclass{b}{A^2} \subseteq \eqvclass{a}{A^2}.
                \]
                Siano dunque \((a, b) \in \rel{A}\) e sia \(c \in \eqvclass{a}{A^2}\), quindi si ha che \((a, v) \in \rel{As} \land (a, b) \in \rel{A}\), dato che \(\rel{A}\) è simmetrica si ha che \((b, a) \in \rel{A} \land (a, c) \in \rel{A}\). Poiché \(\rel{A}\) è transitiva \((b, c) \in \rel{A}\) e dunque \(c \in \eqvclass{b}{A^2}\).
                È stato preso \(c \in \eqvclass{a}{A^2}\) ed è stato provato che \(c \in \eqvclass{b}{A^2}\), da cui segue che \(\eqvclass{a}{A^2} \subseteq \eqvclass{b}{A^2}\). L'altra inclusione è analoga.
            \end{lhs}
            \begin{rhs}
                Siano \(a,b \in A\). Bisogna dimostrare che \((a, b) \in \rel{A}\), si supponga che \(\eqvclass{a}{A^2} = \eqvclass{b}{A^2}\). Dunque, \(a \in \eqvclass{a}{A^2} = \eqvclass{b}{A^2}\) e quindi \(a \in \eqvclass{b}{A^2}\). Si ha quindi che \((b, a) \in \rel{A}\)
                e per simmetria anche \((a, b) \in \rel{A}\).
            \end{rhs}
        \item dimostriamo, anche qui, la doppia implicazione:
            \begin{lhs}
                è necessario dimostrare che \(\eqvclass{a}{A^2} \cap \eqvclass{b}{A^2} = \emptyset\). Siano \(a,b \in \rel{A}\), per assurdo, si supponga che \(\eqvclass{a}{A^2} \cap \eqvclass{b}{A^2} \neq \emptyset\). Allora \(\exists c \in \eqvclass{a}{A^2} \cap \eqvclass{b}{A^2}\) e quindi \((a, c) \in \rel{A} \land (b, c) \in \rel{A}\), che per simmetria diventa \((a, c) \in \rel{A} \land (c, b) \in \rel{A}\) che, per transitività diventa \((a, b) \in \rel{A}\), che è assurdo.
            \end{lhs}
            \begin{rhs}
                bisogna dimostrare che \((a, b) \notin \rel{A}\). Siano \(a,b \in \rel{A}\); per assurdo, si supponga che \((a, b) \in \rel{A}\) allora \(b \in \eqvclass{b}{A^2}\) e \(b \in \eqvclass{a}{A^2}\) per definizione di classe d'equivalenza, dunque si ha \(b \in \eqvclass{b}{A^2} \cap \eqvclass{a}{A^2} = \emptyset\), che è assurdo.
            \end{rhs}
    \end{enumerate}
    La dimostrazione risulta quindi conclusa.
\end{proof}

\begin{remark}
    Siano \(A,B\) due insiemi, si ha
    \[
        A \cup B = \set{x \mid x \in A \lor x \in B};
    \]
    siano \(A,B,C\) tre insiemi, si ha:
    \[
        \left(A \cup B\right) \cup C = A \cup \left(B \cup C\right)
                                     = A \cup B \cup C
                                     = \set{x \mid x \in A \lor x \in B \lor x \in C}.
    \]
    Siano \(A_1, \dotsc, A_n\) insiemi, si ha:
    \begin{alignat*}{1}
        A_1 \cup A_2 \cup \cdots \cup A_n &= \set{x \mid x \in A_1 \lor x \in A_2 \lor \cdots \lor x \in A_n}\\
                                          &= \set{x \mid \exists i = 1, \dotsc, n : x \in A_i}
    \end{alignat*}
    In generale, data una famiglia di insiemi \(A_i\) con \(i \in I \subseteq \R\), si ha:
    \[
        \bigcup_{i \in I} A_i = \set{x \mid \exists i \in I : x \in A_i}
    \]
    e
    \[
        \forall j \in I \quad A_j \subseteq \bigcup_{i \in I} A_i
    \]
\end{remark}

\begin{proposition}%[]\label{prop:}
    Siano \(A\) un insieme non vuoto e \(\rel{A}\) una relazione di equivalenza su \(A^2\). Allora l'unione delle classi di equivalenza di \(x\) al variare di \(x\) in \(A\) è l'insieme \(A\). In simboli:
    \[
        \bigcup_{x \in A} \eqvclass{x}{A^2} = A.
    \]
\end{proposition}

\begin{proof}
    bisogna dimostrare che
    \[
        \bigcup_{x \in A}\eqvclass{x}{A^2} \subseteq A
        \land
        A \subseteq \bigcup_{x \in A}\eqvclass{x}{A^2}.
    \]
    Sia dunque
    \[
        y \in \bigcup_{x \in A}\eqvclass{x}{A^2},
    \]
    allora \(\exists x_0 \in A : y \in \eqvclass{x_0}{A^2} \subseteq A\) e quindi \(y \in A\). Sia ora  \(y \in A\), allora \(y \in \eqvclass{y}{A^2}\) e quindi
    \[
        y \in \bigcup_{x \in A}\eqvclass{x}{A^2}.
    \]
    La dimostrazione risulta conclusa.
\end{proof}


\begin{definition}[partizione]\label{def:partizione}
    Sia \(A\) un insieme non vuoto e sia \(\pi \subseteq \powerset(A)\). si dice che \(\pi\) è una partizione di \(A\) se:
    \begin{enumerate}
        \item \(\forall X \in \pi \quad X \neq \emptyset\);
        \item \(\forall X,Y \in \pi \quad X \neq Y \implies X \cap Y = \emptyset\);
        \item \(\bigcup_{x \in \pi} x = A\).
    \end{enumerate}
\end{definition}


\begin{remark}
    Siano \(A\) un insieme e \(\rel{A}\) una relazione di equivalenza su \(A^2\). Allora l'insieme
    \[
        \pi = \set{\eqvclass{a}{A^2} \;\middle|\; a \in A}
    \]
    è una partizione di \(A\). Infatti
    \begin{enumerate}
        \item \(\forall a \in A \quad \eqvclass{a}{A^2} \neq \emptyset\);
        \item \(\forall a,b \in A \quad \eqvclass{a}{A^2} \neq \eqvclass{b}{A^2} \implies \eqvclass{a}{A^2} \cup \eqvclass{b}{A^2} = \emptyset\);
        \item \(\bigcup_{a \in A} \eqvclass{a}{A^2} = A\).
    \end{enumerate}
\end{remark}

In generale, se \(\pi\) è una partizione su un insieme \(X\) ad essa si può associare una relazione di equivalenza.

\begin{definition}[insieme quoziente]\label{def:insieme_quoziente}
    Siano \(A\) un insieme non vuoto e \(\rel{A}\) una relazione di equivalenza su \(A^2\). Si dice insieme quoziente di \(A\) per \(\rel{A}\) e si indica con il simbolo
    \[
        \quot{A}{\rel{A}} \subseteq \powerset(A)
    \]
    l'insieme di tutte le classi d equivalenza rispetto a \(\rel{A}\). In simboli
    \[
        \quot{A}{\rel{A}} = \set{\eqvclass{a}{A^2} \;\middle|\; a \in \rel{A}}.
    \]
    Si tratta di una partizione su \(A\).
\end{definition}


\section{Funzioni}\label{sec:funzioni}

\begin{definition}[relazione funzionale]\label{def:relazione_funzionale}
    Siano \(A\) e \(B\) due insiemi non vuoti e sia \(\rel{A}{B} \subseteq A \times B\). \(\rel{A}{B}\) è una relazione funzionale se
    \[
        \forall a \in A \quad \exists! b \in B : (a, b) \in \rel{A}{B}.
    \]
\end{definition}

\begin{remark}
    se \(\exists a \in A\) e \(\exists b_1,b_2 \in B\), con \(b_1 \neq b_2\) si può avere che \((a, b_1) \in \rel{A}{B}\) ma anche \((a, b_2) \in \rel{A}{B}\) e quindi \(\rel{A}{B}\) non è più una relazione funzionale.

    Se \(\exists a \in A\) che non è la prima coordinata di alcuna coppia, allora \(\rell\) non è funzionale.
\end{remark}

\begin{example}
    Dati gli insiemi \(A = \set{1, 2, 3, 4, 5}\) e \(B = \set{a, b, x, y}\), la relazione
    \[
        \rel[1]{A}{B} = \set{(1, a), (1, b), (2, a), (3, b), (4, x), (5, x)}
    \]
    non è funzionale perché \((1, a), (1, b) \in \rel[1]{A}{B}\). Al contrario la relazione
    \[
        \rel[2]{A}{B} = \set{(1, a), (2, a), (3, b), (4, x), (5, y)}
    \]
    è funzionale. Invece, la relazione
    \[
        \rel[3]{A}{B} = \set{(1, y), (3, x), (5, a), (4, b)}
    \]
    non è funzionale in quanto non esiste in \(\rel[3]{A}{B}\) alcuna coppia con prima coordinata \(2\).
\end{example}

\begin{definition}[funzione]\label{def:funzione}
    Siano \(A,B\) due insiemi non vuoti e sia \(\rel{A}{B}\) (per brevità
    \(\rell\)) una relazione funzionale tra gli elementi di \(A\) e quelli di
    \(B\). La terna \(f = (A, B, \rell)\) si dice \em{funzione} o
    \em{applicazione} oppure \em{mappa} tra \(A\) e \(B\).

    \begin{itemize}
        \item[\(A\)] si dice insieme di partenza o dominio di \(f\);
        \item[\(B\)] si dice insieme di arrivo di \(f\);
        \item[\(\rell\)] si dice grafico di \(f\).
    \end{itemize}
    La funzione \(f = (A, B, \rell)\) si denota col simbolo
    \[
        f\colon A \to B
    \]
    Quindi
    \[
        \forall a \in A \quad \exists! b \in B \tc (a, b) \in \rell.
    \]
\end{definition}

\begin{examples}
    Seguono degli esempi:
    \begin{enumerate}
        \item \(\rell{1} = \set{(x, y) \in \N \times \Z \mid y = -x}\) è una relazione funzionale, infatti
            \[
                \forall x \in \N \;\; \exists! y = -x \in \Z \tc (x, y) \in \R
            \]
            Scriviamo la funzione come \(f_1 = (\N, \Z, \rell{1})\) e dunque \(\forall x \in \Z \; f(x) = -x\).
            \(f_1\) si può scrivere anche come \(f_1 \colon \N \to \Z\) e si ha quindi \(x \mapsto -x\);
%
        \item \(\rell{2} = \set{(x, y) \in \Z \times \Z \mid y = -x}\) è una relazione funzionale. \\
            Scriviamo la funzione come \(f_2 = (\Z, \Z, \R)\), essa si può quindi scrivere anche come \(f_2 = \Z \to \Z\) e quindi si ha \(x \mapsto -x\);
%
        \item \(\rell{3} = \set{(x, y) \in \Q^* \times \Q \mid y = 2/x}\) è una relazione funzionale, infatti
            \[
                \forall x \in \Q^* \quad \exists! y \in \Q \tc y = 2/x
            \]
            Scriviamo la funzione come \(f_3 \colon \Q^* \to \Q\) e quindi \(x \mapsto 2/x\);
%
        \item \(\rell{4} = \set{(x, y) \in \Q \times \Q \mid y = 2/x}\) non è una relazione funzionale, perché
            \[
                \exists x = 0 \in \Q \tc \forall y \in \Q \quad (0, y) \notin \rell{4};
            \]
%
        \item \(\rell{5} = \set{(x, y) \in \Q^* \times \Q^* \mid y = 2/x}\) è una relazione funzionale. Scriviamo la funzione come \(f_5 = (\Q^*, \Q^*, \rell{5})\) oppure come \(f_5 \colon \Q^* \to \Q^*\) e dunque \(x \mapsto 2/x\).
    \end{enumerate}
\end{examples}


\begin{remark}
    Siano \(f \colon A \to B\) e \(g \colon C \to D\) due funzioni, segue che \(x \mapsto f(x)\) e \(y \mapsto g(y)\). Si ha che
    \[
        f = g \Longleftrightarrow A = C \land B = D \land \forall x \in A = C \quad f(x) = g(x).
    \]
\end{remark}
