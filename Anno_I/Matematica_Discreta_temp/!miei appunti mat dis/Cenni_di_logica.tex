% !TEX root = ./Matematica discreta.tex

\chapter{Cenni di Logica}\label{chapt:cenni_di_logica}

\section{Proposizioni Atomiche}\label{sec:proposizioni_atomiche}
Le proposizioni possono essere distinte in:

\begin{itemize}
    \item proposizioni \textbf{atomiche}, dette anche proposizioni semplici;
    \item proposizioni \textbf{molecolari}, dette anche proposizioni composte.
\end{itemize}

Una proposizione si dice \textbf{atomica} o semplice se è formata da un
soggetto, da un verbo ed eventualmente da uno o più complementi. In altre
parole la proposizione atomica non può essere scomposta in parti più semplici.
Queste proposizioni permettono di dire con certezza se siano Vere (\true[l]) o
False (\false[l])
Alcuni esempi sono i seguenti:

\begin{enumerate}[label=\(\alph*\))]
    \item \(5\) è un numero primo (\true[l]);
    \item \(10\) è un numero dispari (\false[l]);
    \item Roma si trova in Piemonte (\false[l]).
\end{enumerate}

Proposizioni come ``la matematica è bella'' non permettono di dire se siano
Vere o False.

Si prenda la proposizione ``\(x\) è un numero positivo''. Questa non è una
proposizione atomica e non è possibile dire se sia \true[l]{} o \false[l]{}
poiché è presente un'incognita, ovvero \(x\); in base all'insieme di
appartenenza di \(x\) il valore (\true[l]/\false[l]) della proposizione
cambia. È necessario sostituire un numero al posto di \(x\), sostituendo
\(3/2\) la proposizione risulta \true[l]{}, sostituendo \(-\sqrt{2}\) risulta
\false[l]{}.

Per poter conoscere il valore della proposizione precedente è possibile
utilizzare dei quantificatori:

\begin{itemize}
    \item \(\forall\) – ``per ogni'';
    \item \(\exists\) – ``esiste''.
\end{itemize}

Questi quantificatori hanno senso se la variabile varia in un universo,
quest'ultimo non è altro che un insieme.

Un insieme in Matematica è un raggruppamento di elementi di qualsiasi tipo
numerico, logico o concettuale) che può essere individuato mediante una
caratteristica comune agli elementi che vi appartengono oppure per semplice
elencazione degli elementi dell'insieme. L'insieme dei numeri naturali \((\N\),
ad esempio, racchiude al suo interno tutti i numeri interi positivi. Gli
oggetti di un insieme vengono detti \textbf{elementi}. L'argomento verrà
trattato approfonditamente nel capitolo~\nameref{chapt:Cenni_Teoria_Insiemi}.

Se \(A\) è un insieme e \(a\) è un suo elemento è possibile scrivere

\[
    a \in A \quad \text{oppure} \quad A \ni a
\]

che si legge ``\(a\) appartiene ad \(A\)'' oppure ``\(a\) è elemento di \(A\)''.
Al contrario, per esprimere la non appartenenza all'insieme si scriverà

\[
    a \notin A \quad \text{oppure} \quad A \centernot\ni a
\]

\begin{example}
    Si prenda come universo l'insieme \(\N\) che è definito come
    \(\N \coloneqq \{0,1,2,\dotsc,n\}\).
    La proposizione \(1 \in \N\) risulta essere \true[l]. Prendendo la
    proposizione \(\forall x \in \N \; x \geq 0\) essa risulta \true{}
    considerando lo \(0\) positività).

    Si prenda ora l'insieme \(\Z \coloneqq \{\dotsc,-2,-1,0,1,2,\dotsc\}\), l'insieme dei numeri relativi. Prendendo la proposizione \(\forall x \in \Z \; x \geq 0\) essa risulterà \false{}. Scrivendo invece \(\exists x \in \Z \tc x \geq 0\) la proposizione risulta \true[l].
\end{example}


\section{Connettivi Logici}\label{sec:connettivi_logici}
Un connettivo logico o operatore logico, è un elemento grammaticale di collegamento che instaura fra due proposizioni \(a\) e \(b\) una qualche relazione che dia origine ad una terza proposizione \(c\) con un valore vero o falso, in base ai valori delle due proposizioni 
% fattori 
ed al carattere del connettivo utilizzato. I connettivi logici permettono dunque di formare delle proposizioni più complesse.

\subsection{Negazione}\label{subsec:negazione}
Il primo connettivo logico è la \textsc{negazione}: data una proposizione atomica \(a\), la sua negazione si indica con \(\neg{a}\) oppure con \(\bar{a}\) e si legge ``non \(a\)''. Dunque \(\neg{a}\) è negazione di \(a\), di conseguenza se \(a\) è \true[l]{} allora \(\neg{a}\) è \false[l]{} e, al contrario, se \(a\) è \false[l]{} allora \(\neg{a}\) è \true[l]{}. Se una proposizione è vera le si assegna il valore \(1\), se è falsa le si assegna il valore \(0\).

% !TEX root = ../Matematica discreta.tex

\begin{table}[h]
\centering
    \caption{%
        Tabella di verità della \textsc{negazione} (\(\neg\))
    }\label{tab:T-F_negazione}
%
    \begin{minipage}[t]{0.5\textwidth}
\centering
        \begin{tabular}[t]{ c c }        \toprule
            \(a\)        & \(\neg{a}\)   \\ \midrule
            \true[l]{}   & \false[l]{}      \\
            \false[l]{}  & \true[l]{}       \\ \bottomrule
        \end{tabular}
    \end{minipage}%
\hspace*{-3cm}
    \begin{minipage}[t]{0.5\textwidth}
\centering
        \begin{tabular}[t]{ c c }          \toprule
            \(a\)       & \(\neg{a}\)   \\ \midrule
            \true[b]{}  & \false[b]{}   \\
            \false[b]{} & \true[b]{}    \\ \bottomrule        
        \end{tabular}
    \end{minipage}
\end{table}

\begin{example}
    Si prenda la proposizione \(b\): ``\(10\) è un numero dispari'' che ha valore \false, la sua negazione \(\neg b\) sarà ``\(10\) \textbf{non} è un numero dispari e assume valore'' \true.
\end{example}

\begin{example}
    Si prenda la proposizione \(p\): ``Milano è la capitale dell'Italia'', essa ha valore \false, però la sua negazione \(\neg p\), ``Milano non è la capitale dell'Italia'', ha valore \true.
\end{example}


\subsection{Disgiunzione}\label{subsec:disgiunzione}
Il secondo connettivo logico è la \textsc{disgiunzione}, essa si indica col simbolo ``\(\lor\)''. Prese due proposizioni \(a, b\) la proposizione \(a \lor b\) si legge ``\(a\) \textbf{o} \(b\)'' e si chiama disgiunzione di \(a\) e \(b\).

% !TEX root = ../Matematica discreta.tex

\begin{table}[h]
\centering
\caption{Tabella della verità della \textsc{disgiunzione} (\(\lor\))}\label{tab:T-F_disgiunzione}
    \begin{tabular}[t]{ c c c }                               \toprule
        \(a\) & \(b\) & \(a \lor b\)                       \\ \midrule
          \true[b]{}   &   \true[b]{}    &   \true[b]{}    \\
          \true[b]{}   &   \false[b]{}   &   \true[b]{}    \\
          \false[b]{}  &   \true[b]{}    &   \true[b]{}    \\
          \false[b]{}  &   \false[b]{}   &   \false[b]{}   \\ \bottomrule
    \end{tabular}
\end{table}

\begin{example}
    Si prendano le proposizioni \(p\): ``\(\sqrt{2}\) è un numero razionale'' (\false) e \(q\): ``\(5\) è un numero pari'' (\false). Si avrà che \(p \lor q\): ``(\(\sqrt{2}\) è un numero razionale) \(\lor\) (\(5\) è un numero pari)'' risulterà essere Falsa (\false).

    Preso \(q'\): ``\(5\) è un numero dispari'', si ha che \(p \lor q'\): ``(\(\sqrt{2}\) è un numero razionale) \(\lor\) (\(5\) è un numero dispari)'' che ha valore di verità (\true).
\end{example}


\subsection{Congiunzione}\label{subsec:congiunzione}
Il terzo connettivo logico è la \textsc{congiunzione}, essa si indica con il simbolo ``(\(\land\))''. Prese due proposizioni atomiche \(a\) e \(b\), allora la proposizione \(a \land b\) si legge ``\(a\) \textbf{e} \(b\)'' si dice congiunzione di \(a\) e \(b\).

% !TEX root = ../Matematica discreta.tex

\begin{table}[h]
\centering
\caption{Tabella della verità della \textsc{congiunzione} (\(\land\))}\label{tab:T-F_congiunzione}
    \begin{tabular}[t]{ c c c }                                \toprule
        \(a\) & \(b\) & \(a \land b\)                       \\ \midrule
          \true[b]{}    &   \true[b]{}    &   \true[b]{}    \\
          \true[b]{}    &   \false[b]{}   &   \false[b]{}   \\
          \false[b]{}   &   \true[b]{}    &   \false[b]{}   \\
          \false[b]{}   &   \false[b]{}   &   \false[b]{}   \\ \bottomrule
    \end{tabular}
\end{table}

\begin{example}
    Si prendano le proposizioni \(p\): ``la mosca è un insetto'' (\true) e \(q\): ``\(4\) è un multiplo di \(2\)'' (\true). La proposizione \(p \land q\): ``(la mosca è un insetto) \(\land\) (\(4\) è un multiplo di \(2\))'' (\true).
\end{example}

\begin{example}
    Si prendano le proposizioni \(r\): ``l'asino vola'' (\false) e \(s\) ``\(\sqrt{2}\) è positivo'' (\true). La proposizione \(r \land s\): ``(l'asino vola) \(\land\) (\(\sqrt{2}\) è positivo)'' risulta essere falsa (\false)
\end{example}


\subsection{Implicazione}\label{subsec:implicazione}
Il quarto connettivo logico è l'\textsc{implicazione}, essa si indica con il simbolo ``\(\longrightarrow\)''. Prese due proposizioni atomiche \(a\) e \(b\), allora la proposizione \(a \longrightarrow b\) si legge ``\(a\) implica \(b\)'' oppure ``se \(a\) allora \(b\)'' e presenta la seguente tabella di verità

% !TEX root = ../Matematica discreta.tex

\begin{table}[h]
\centering
\caption{Tabella di verità dell'\textsc{implicazione} (\(\longrightarrow\))}\label{tab:T-F_implicazione}

    \begin{tabular}[t]{ c c c }                                  \toprule
        \(a\)         &   \(b\)       & \(a \longrightarrow b\)   \\ \midrule
        \true[b]{}    & \true[b]{}    &   \true[b]{}          \\
        \true[b]{}    & \false[b]{}   &   \false[b]{}         \\
        \false[b]{}   & \true[b]{}    &   \true[b]{}          \\
        \false[b]{}   & \false[b]{}   &   \true[b]{}          \\ \bottomrule
    
    \end{tabular}
\end{table}

\begin{paradosso_russel}
    Se \(1 = 2\) allora io sono il Papa. Se \(1 = 2\) allora io e il Papa siamo due, ma \(2 = 1\) allora io sono il Papa.
\end{paradosso_russel}
Il paradosso si verifica poiché si parte da una proposizione falsa.

\begin{example}
    Prese le proposizioni \(a\): ``l'automobile viaggia'' e \(b\): ``l'automobile ha carburante''. Ne risulta che se la \(a\) ha valore di verità \true, allora \(b\) ha valore di verità \true{}; se, invece, \(a\) ha valore di verità \false, allora non è possibile dire che \(b\) sia vera.
\end{example}


\subsection{Doppia implicazione}\label{subsec:doppia_implicazione}
Il quinto connettivo logico è la \textsc{doppia implicazione}, essa si indica con ``\(\longleftrightarrow \)''. Prese due proposizioni atomiche \(a\) e \(b\), allora la proposizione \(a \longleftrightarrow  b\) si legge ``\(a\) se e solo se \(b\)'' e corrisponde a \((a \longrightarrow b) \land (b \longrightarrow a)\). Ha la seguente tabella di verità.

% !TEX root = ../Matematica discreta.tex

\begin{table}[h]
\centering
\caption{Tabella di verità della \textsc{doppia implicazione} (\(\longleftrightarrow\))}\label{tab:T-F_doppia_implicazione}
\hspace*{1cm}
    \begin{minipage}[t]{0.5\textwidth}
    \centering
        \begin{tabular}[t]{ c c c c c }                                                      \toprule
            \(a\) & \(b\)
                & \(a \longrightarrow b\) & \(b \longrightarrow a\)
                    & \((a \longrightarrow b) \land (b\longrightarrow a)\)                        \\ \midrule
            \true[b]{}    & \true[b]{}    & \true[b]{}    & \true[b]{}    & \true[b]{}    \\
            \true[b]{}    & \false[b]{}   & \false[b]{}   & \true[b]{}    & \false[b]{}   \\
            \false[b]{}   & \true[b]{}    & \true[b]{}    & \false[b]{}   & \false[b]{}   \\
            \false[b]{}   & \false[b]{}   & \true[b]{}    & \true[b]{}    & \true[b]{}    \\ \bottomrule 
        \end{tabular}%
    \end{minipage}%
\hspace*{-1cm}%
    \begin{minipage}[t]{0.5\textwidth}
    \centering
        \begin{tabular}[t]{ c c c }                          \toprule
            \(a\) & \(b\) & \(a \longleftrightarrow b\)   \\ \midrule
            \true[b]{}    & \true[b]{}   & \true[b]{}     \\
            \true[b]{}    & \false[b]{}  & \false[b]{}    \\
            \false[b]{}   & \true[b]{}   & \false[b]{}    \\
            \false[b]{}   & \false[b]{}  & \true[b]{}     \\ \bottomrule 
        \end{tabular}     
    \end{minipage}%
\end{table}


\section{Formule della logica proposizionale}\label{sec:formule_logica_proposizionale}
Siano \(a_1, a_2, \dotsc, a_n\) dei simboli. Una formula si ottiene nel modo seguente:
\begin{enumerate}
    \item \(a_1, a_2, \dotsc, a_n\) sono formule;
    \item se \(a,b\) sono formule, sono formule anche:
        \[
            \neg a, \quad a \land b, \quad a \lor b, \quad a \longrightarrow b, \quad a \longleftrightarrow  b
        \]
    \item le formule si ottengono esclusivamente da \(1\). e \(2\).
\end{enumerate}

\begin{example}
    Siano \(p, q, r\) delle formule allora anche
    \begin{gather}
        \big(p \land (q \longrightarrow r)\big) \longleftrightarrow \big((\neg q) \lor (\neg r)\big) \label{eq:formula_1} \\
        \big((\neg p) \longrightarrow (q \land r)\big) \lor \big(p \longrightarrow q\big) \label{eq:formula_2}    
    \end{gather}
\end{example}
risultano essere delle formule. L'ordine da seguire è il seguente:
\begin{enumerate}
    \item negazione \((\neg)\);
    \item congiunzione e disgiunzione \((\land, \lor)\);
    \item implicazione e doppia implicazione \((\longrightarrow, \longleftrightarrow)\).  
\end{enumerate}
È dunque possibile eliminare delle parentesi che risultano essere in eccesso, l'\ref{eq:formula_1} si scriverà:
\[
    \big(p \land (q \longrightarrow r)\big) \longleftrightarrow (\neg q \lor \neg r)
\]
e l'\ref{eq:formula_2} diventa:
\[
    (\neg p \longrightarrow q \land r) \lor (p \longrightarrow q).
\]
La tavola di verità, ad esempio, dell'\ref{eq:formula_1} è la~\cref{tab:T-F_formula_1}.\\

% !TEX root=../Matematica discreta.tex

\begin{table}[h]
\centering
\caption{Tabella di verità dell'\ref{eq:formula_1}}\label{tab:T-F_formula_1}
\begin{tabular}[t]{ c c c c c c c c c } \toprule
    \(p\) & \(q\) & \(r\)
        & \(p \longrightarrow r\) & \(p \land (q \longrightarrow r)\)
            & \(\neg q\) & \(\neg r\) & \(\neg q \lor \neg r\)
                & \(\big(p \land (q \longrightarrow r)\big) \longleftrightarrow (\neg q \lor \neg r)\) \\ \midrule
    \true[b]  &\true[b]  &\true[b]  &\true[b]  &\true[b]  &\false[b] &\false[b] &\false[b] &\false[b] \\
    \true[b]  &\true[b]  &\false[b] &\false[b] &\false[b] &\false[b] &\true[b]  &\true[b]  &\false[b] \\
    \true[b]  &\false[b] &\true[b]  &\true[b]  &\true[b]  &\true[b]  &\false[b] &\true[b]  &\true[b]  \\
    \true[b]  &\false[b] &\false[b] &\true[b]  &\true[b]  &\true[b]  &\true[b]  &\true[b]  &\true[b]  \\
    \false[b] &\true[b]  &\true[b]  &\true[b]  &\false[b] &\false[b] &\false[b] &\false[b] &\true[b]  \\
    \false[b] &\true[b]  &\false[b] &\false[b] &\false[b] &\false[b] &\true[b]  &\true[b]  &\false[b] \\
    \false[b] &\false[b] &\true[b]  &\true[b]  &\false[b] &\true[b]  &\false[b] &\true[b]  &\false[b] \\
    \false[b] &\false[b] &\false[b] &\true[b]  &\false[b] &\true[b]  &\true[b]  &\true[b]  &\false[b] \\ \bottomrule
\end{tabular}
\end{table}

\noindent Nel caso di una formula con \(k\) variabili si esamineranno \(2^k\) casi differenti.\\

Se la tavola di verità di una formula è sempre vera, allora tale formula si dice \underline{tautologia}; se è sempre falsa si dice \underline{contraddizione}.
\newpage

\begin{example}
    Sia \(a\) una formula. Si avrà che \(a \land \neg a\) è una contraddizione, infatti:
    % !TEX root = ../Matematica discreta.tex

\begin{table}[h]
\centering
    \begin{tabular}[t]{ c c c }                          \toprule
        \(a\)     & \(\neg a\) & \(a \land \neg a\)   \\ \midrule
        \true[b]  & \false[b]  & \false[b]            \\
        \false[b] & \true[b]   & \false[b]            \\ \bottomrule
    \end{tabular}
\end{table}
    \noindent Al contrario, \(\neg a \lor a\) è una tautologia, infatti:
    %  !TEX root = ../Matematica discreta.tex

\begin{table}[ht]
\centering
    \begin{tabular}[t]{ c c c }                     \toprule
        \(a\) & \(\neg a\) & \(a \lor \neg a\)   \\ \midrule
        \true[b]  & \false[b]  & \true[b]        \\
        \false[b] & \true[b]   & \true[b]        \\ \bottomrule
    \end{tabular}
\end{table}
\end{example}

\begin{example}
    Siano \(a, b\) due formule. Un altro esempio di tautologia è la seguente formula
    \[
        \neg(a \land b) \longleftrightarrow \neg a \lor \neg b,
    \]
    la tabella di verità è la seguente:
    % !TEX root = ../Matematica discreta.tex

\begin{table}[h!]
\centering
    \begin{tabular}[t]{ c c c c c c c c }  \toprule
        \(a\) & \(b\) & \(a \land b\) & \(\neg (a \land b)\)
            & \(\neg a\) & \(\neg b\) & \(\neg a \lor \neg b\)
                & \(\neg(a \land b) \longleftrightarrow \neg a \lor \neg b\) \\ \midrule
        \true[b]  & \true[b]  & \true[b]  & \false[b] & \false[b] & \false[b] & \false[b] & \true[b] \\
        \true[b]  & \false[b] & \false[b] & \true[b]  & \false[b] & \true[b]  & \true[b]  & \true[b] \\
        \false[b] & \true[b]  & \false[b] & \true[b]  & \true[b]  & \false[b] & \true[b]  & \true[b] \\
        \false[b] & \false[b] & \false[b] & \true[b]  & \true[b]  & \true[b]  & \true[b]  & \true[b] \\ \bottomrule
    \end{tabular}

\end{table}
\end{example}
% \newpage

\begin{example}
    Siano \(a\): ``\(5\) è pari'' e \(b\): ``\(3\) è primo'' due proposizioni. Si avrà che:
    \begin{itemize}
        \item \(a \land b\) è Falsa (\false);
        \item \(\neg(a \land b)\) è Vera (\true);
        \item \(\neg(a \land b) \longleftrightarrow \neg a \lor \neg b \implies (5 \text{ è dispari}) \lor (3 \text{ non è primo})\) è Vera (\true).
    \end{itemize}
\end{example}

\begin{definition}[conseguenza logica]\label{def:implicazione_logica}
    Siano \(a, b\) due formule. Si dice che \(b\) è conseguenza logica di \(a\) e si scrive
    \[
        a \implies b
    \]
    se \(b\) è vero ogni qualvolta \(a\) è vera.
\end{definition}

\begin{definition}[formule semanticamente equivalenti]\label{def:semanticamente_equivalenti}
    Siano \(a, b\) due formule. Si dice che \(a\) e \(b\) sono semanticamente equivalenti se \(b\) è conseguenza logica di \(a\) e \(a\) è conseguenza logica di \(b\):
    \[
        a \iff b \quad\text{vuol dire}\quad [a \implies b] \land [b \implies a].
    \]
\end{definition}

\begin{proposition}\label{prop:stessa_tab_T-F}
    Siano \(a, b\) due formule. Risulta che \(a\) e \(b\) sono semanticamente equivalenti se e soltanto se hanno la stessa tavola di verità.
\end{proposition}

\begin{proof}
    \begin{lhs}
        Supponiamo che \(a\) e \(b\) siano semanticamente equivalenti. Se \(a\) ha valore di verità \true, allora anche \(b\) ha valore di verità \true, perché \(b\) è conseguenza logica di \(a\). Se \(a\) ha valore di verità \false, allora anche \(b\) ha valore di verità \false, perché se \(b\) avesse valore di verità \true, \(a\), che è conseguenza logica di \(b\), avrebbe valore di verità \true. Scambiando tra loro \(a\) e \(b\) si deduce che \(a\) e \(b\) hanno la stessa tavola di verità.
    \end{lhs}
    \begin{rhs}
        Supponiamo che \(a\) e \(b\) abbiano la stessa tavola di verità. Se \(a\) ha valore di verità \true, allora anche \(b\) ha valore di verità \true{} e quindi \(b\) è conseguenza logica di \(a\). Se \(b\) ha valore di verità \true{}, allora anche \(a\) ha valore di verità \true{} e quindi \(a\) è conseguenza logica di \(b\) e dunque \(a\) e \(b\) sono semanticamente equivalenti.
    \end{rhs}
\end{proof}

\begin{remark}
    Siano \(a, b\) due formule. Allora \(a \Longleftrightarrow b\) se e solo se \(a \longleftrightarrow b\) è una tautologia.\\
\end{remark}


\section{Regole di inferenza}\label{sec:regole_di_inferenza}
Nella logica matematica una regola di inferenza è uno schema formale che si applica nell'eseguire un'inferenza. In altre parole, è una regola che permette di passare da un numero finito di proposizioni assunte come premesse a una proposizione che funge da conclusione.

\subsection{Modus Ponens o Metodo di Dimostrazione Diretta}\label{subsec:modus_ponens}
Nella logica, il modus ponens, è una semplice e valida regola d'inferenza, che afferma che se \(P \longrightarrow  Q\) è una proposizione Vera, e anche la premessa \(P\) è Vera, allora la conseguenza \(Q\) è vera; in notazione con operatori logici:
\[
    \big(P \land (P \longrightarrow Q)\big) \implies Q.
\]
La conclusione si evince dalla tabella di verità:
% !TEX root = ../Matematica discreta.tex

\begin{table}[h]
\centering
    \begin{tabular}[t]{ c c c }                  \toprule
        \(P\) & \(Q\) & \(P \longrightarrow Q\)   \\ \midrule
        \true  & \true  & \true               \\
        \true  & \false & \false              \\
        \false & \true  & \true               \\
        \false & \false & \true               \\ \bottomrule
    \end{tabular}
\end{table}

\noindent infatti quando \(P\) e \(P \longrightarrow Q\) sono Vere, anche \(Q\) è vero.

\begin{example}
    Se \(n\) è un numero intero pari, allora anche \(n^2\) è un numero intero pari.
    Dunque la proposizione \(P\): ``\(n\) è pari'' afferma che \(\exists h \tc n = 2h\). Verifichiamo che la proposizione \(Q\): ``\(n^2\) è pari'' sia Vera:
    \[
        n^2 = {\left(2h\right)}^2
            = 2^2 h^2
            = 2 \left(2h^2\right)
    \]
    che è effettivamente un numero pari dato che \(\exists k = 2h^2 \tc n^2 = 2k\).
    % Si noti che se \(P\) è Vera e \(P \longrightarrow Q\) è Vera allora anche \(Q\) è vera.
\end{example}


\subsection{Modus Tollens}\label{subsec:modus_tollens}
Siano \(P, Q\) due formule. Il modus tollens asserisce che se \(P \longrightarrow Q\) è Vera, ed è Vera anche la negazione di \(Q\), allora la negazione di \(P\) è un enunciato vero; in simboli:
\[
    \left[(P \longrightarrow Q) \land \neg Q\right] \longrightarrow \neg P
\]
La tabella di verità è la seguente:
% !TEX root = ../Matematica discreta.tex

\begin{table}[h]
\centering
    \begin{tabular}[t]{ c c c c c c c }                                            \toprule
        \(P\) & \(Q\)
            & \(\neg P\) & \(\neg Q\)
                & \(P \longrightarrow Q\) & \((P \longrightarrow Q) \land \neg Q\)
                    & \([(P \longrightarrow Q) \land \neg Q] \longrightarrow \neg P\)   \\ \midrule

        \true  & \true  & \false & \false & \true  & \false & \true             \\ 
        \true  & \false & \false & \true  & \false & \false & \true             \\ 
        \false & \true  & \true  & \false & \true  & \false & \true             \\ 
        \false & \false & \true  & \true  & \true  & \true  & \true             \\ \bottomrule
    \end{tabular}
\end{table}
Grazie alla tavola di verità del modus tollens possiamo concludere che esso è una tautologia.
\begin{example}
    Si considerino le due proposizioni \(P\): ``Luca ha sete'' e \(Q\): ``Luca beve''. Le loro negazioni sono \(\neg P\): ``Luca non ha sete'' e \(\neg Q\): ``Luca non beve''. La proposizione \(P \longrightarrow Q\) equivale a: ``Luca ha sete allora Luca beve''. Secondo la regola del modus tollens, se \(P \longrightarrow Q \) è Vera e anche \(\neg Q\) è Vera, allora \(\neg P\) è Vera. In questo esempio: ``se Luca ha sete allora beve (\(P \longrightarrow Q\)), ma Luca non beve (\(\neg Q\)), quindi Luca non ha sete (\(\neg P\))''.
\end{example}

\begin{remark}
    Si ricordi che il modus tollens afferma che se \(P \longrightarrow Q\) è un enunciato Vero, ed è Vera anche la negazione di \(Q\), allora \(\neg P\) è un enunciato Vero.
    La sua corretta formulazione con i simboli dovrebbe essere la seguente:
    \[
        [(P \longrightarrow Q) \land \neg Q] \implies \neg P,
    \]
    ossia l'ultima implicazione dovrebbe essere un'implicazione logica. Tuttavia, poiché il modus tollens è una tautologia, i due simboli possono essere usati indistintamente.
\end{remark}


\subsection{Dimostrazione per contrapposizione}\label{subsec:dim_contrapposizione}
In matematica, la dimostrazione per contrapposizione è una regola di inferenza in cui si deduce la proposizione dalla sua contropositiva. In altre parole, la conclusione \(P \longrightarrow Q\) è dedotta dimostrando che \(\neg Q \longrightarrow \neg P\). Spesso questo approccio viene utilizzato quando la negazione è più semplice da dimostrare rispetto alla proposizione originale. In formule:
\[
    [P \longrightarrow Q] \iff [\neg Q \longrightarrow \neg P].
\]

Logicamente, la validità della dimostrazione per contrapposizione può essere dimostrata osservando la seguente tavola di verità:
% !TEX root = ../Matematica discreta.tex

\begin{table}[h]
\centering
    \begin{tabular}[t]{ c c c c c c }                                      \toprule
        \(P\) & \(Q\)
            & \(\neg P\) & \(\neg Q\)
                & \(P \rightarrow Q\) & \(\neg Q \rightarrow \neg P\)   \\ \midrule
        \true  & \true  & \false & \false & \true  & \true              \\
        \true  & \false & \false & \true  & \false & \false             \\
        \false & \true  & \true  & \false & \true  & \true              \\
        \false & \false & \true  & \true  & \true  & \true              \\ \bottomrule
    \end{tabular}
\end{table}
Si può notare come \(P \longrightarrow Q\) e \(\neg Q \longrightarrow \neg P\) hanno la stessa tavola di verità, dunque per la~\cref{prop:stessa_tab_T-F} sono semanticamente equivalenti.

\begin{example}
    Sia \(n\) un numero intero. Si provi che se \(n^2\) è pari, allora \(n\) è pari. Nonostante una dimostrazione diretta (\cref{subsec:modus_ponens}) è possibile, usiamo la dimostrazione per contrapposizione. Il contropositivo della proposizione precedente è: ``se \(n\) non è pari, \(n^2\) non è pari''. Quest'ultima proposizione può essere provata come segue: se \(n\) non è pari allora è dispari. Il prodotto di due numeri dispari è dispari, quindi \(n^2 = n \cdot n\) è dispari, dunque \(n^2\) non è pari.
    Avendo provato il contropositivo, possiamo affermare che la proposizione iniziale è Vera.
\end{example}


\subsection{Dimostrazione per assurdo}\label{subsec:dimostrazione_per_assurdo}
La dimostrazione per assurdo è un tipo di argomentazione logica nella quale, muovendo dalla negazione della tesi che si intende sostenere e facendone seguire una sequenza di passaggi logico-deduttivi, si giunge a una conclusione incoerente e contraddittoria. Tale risultato, nella logica argomentativa, confermerebbe l'ipotesi iniziale, per mezzo della falsificazione della sua negazione. In formule:
\[
    \big[\neg Q \longrightarrow (R \land \neg R)\big] \implies Q,
\]
dove \(R\) è un'altra proprietà di cui è già noto il valore di verità.

\begin{example}
    Si consideri la proposizione \(Q\): ``non esistono \(x,y\) numeri interi tali che \(3x + 6y = 5\)''. Supponiamo \((\neg Q)\) che esistano due numeri interi \(x_0, y_0\) tali che 
    \[
        3x_0 + 6y_0 = 5
    \]
    allora
    \[
        3(x_0 + 2y_0) = 5
    \]
    dunque \(5\) sarebbe un multiplo di \(3\). Dunque si ha che \(R\): ``\(5\) non è multiplo di \(3\)'' \textbf{e} \(\neg R\): ``\(5\) è multiplo di \(3\)''. Si è giunti ad una contraddizione, dunque la dimostrazione risulta conclusa.
\end{example}


\section{Predicati}\label{sec:predicati}
Si prenda in considerazione \(P\): ``\(x\) è un numero pari'', questa non è una proposizione perché non è possibile sapere il suo valore di verità. Quest'ultimo dipende dal valore che l'incognita \(x\) assume: al variare di \(x\) si otterranno proposizioni che potranno essere vere o false. Una frase del genere è dunque detta \textbf{predicato} e a seconda dei valori della \(x\) si trasformerà in una proposizione Vera o una proposizione Falsa.
È possibile sostituire qualsiasi valore alla \(x\)? No, perché soltanto alcuni valori della \(x\) danno un senso alla frase: ``\(7\) è un numero pari'' è Falso, ma almeno ha senso; ``cerchio è un numero pari'' non è né vero né falso perché la frase non ha alcun significato.

Quindi gli elementi che possono essere sostituiti alla \(x\) fanno parte di un determinato insieme e sono tali che sostituendoli alla \(x\) si ottiene una frase dotata di senso. Tale insieme è detto dominio del predicato. Il dominio è un insieme da cui vengono presi degli elementi da sostituire alla \(x\) e stabilire il valore di verità del predicato.
L'insieme dei valori di \(x\) che rendono vero il predicato si chiama insieme di verità del predicato.

Per esprimere un predicato in formule si scriverà:
\begin{gather*}
    \big(\forall x \in U_x\big) \left(P(x)\right) \\
    \big(\exists x \in U_x\big) \left(P(x)\right)
\end{gather*}
dove \(U_x\) è l'universo in cui varia la variabile \(x\) e \(P(x)\) è una proprietà che ha senso per \emph{tutti} gli elementi di \(U_x\).

\begin{example}
    Si consideri il predicato \(P(x)\): ``\(x\) è pari''. Preso \(U_x = \Z = \{\dotsc, -2,-1,0,1,2, \dotsc\}\). Si ha dunque \(Q\colon \left(\forall x \in \Z\right)(P(x))\), ovvero:
    \[
        Q\colon \left(\forall x \in \Z\right) (x \text{ è pari})
    \]
    che risulta essere una proposizione Falsa perché, ad esempio, \(\exists 5 \in \Z\) che è un numero dispari.
    Preso invece il predicato ``\(\forall x \in \Z\;\;\; x\) è un numero intero'', esso risulta essere Vero.
    Si consideri ora \(R\colon \forall x \in \Z x \) è positivo è Falsa perché, ad esempio, \(-2 \in \Z\) non verifica \(R\).
\end{example}

\noindent Diamo allora la definizione di predicato.
\begin{definition}[predicato]\label{def:predicato}
    Un predicato è un'affermazione che coinvolge una o più variabili: \(x,y,z, \dotsc\) ciascuna delle quali varia in un universo \(U_x,U_y,U_z, \dotsc\) con l'uso di un opportuno quantificatore.
\end{definition}

\begin{example}
    Si prenda in considerazione la proposizione \(A\): ``ogni numero intero relativo moltiplicato per \(1\) dà per risultato lo stesso numero intero'', che in simboli diventa:
    \[
        A\colon \left(\forall a \in \Z\right)(a \cdot 1 = 1 \cdot a = a),
    \]
    dove \(P(a)\) è appunto \((a \cdot 1 = a \cdot a = a)\), che ricordiamo può essere scritto anche come \((a \cdot 1 = a \land 1 \cdot a = a)\). Ovviamente la proposizione \(A\) risulta essere Vera. Dunque \(\neg A\) è Falsa:
    \begin{alignat*}{1}
        \neg A\colon (\exists a \in \Z)\; \neg \left(P(a)\right)%
            &\implies (\exists a \in \Z)\; \neg (a \cdot 1 = a \cdot a = a) \\
            &\implies (\exists a \in \Z)\; [\neg (a \cdot 1 = a) \lor \neg (1 \cdot a = a)] \\
            &\implies (\exists a \in \Z)\; (a \cdot 1 \neq a \lor 1 \cdot a \neq a);
    \end{alignat*}
    questo perché:
    \[
        \neg (a \land b) \iff \neg a \land \neg b
    \]
\end{example}

\begin{example}
    Si consideri la proposizione \(B\): ``ogni numero intero naturale è dispari''. Vediamo innanzitutto come si definisce un numero dispari in \(\Z\) e in \(\N\):
    \begin{alignat*}{1}
        n \in \Z \text{ dispari se } &\exists h \in Z \tc n = 2h + 1, \\
        n \in \N \text{ dispari se } &\exists h \in Z \tc n = 2h + 1.
    \end{alignat*}
    Per esempio \(-15\) è dispari, infatti \(\exists h = -8 \in \Z \tc {-}15 = 2(-8) + 1\).

    Scriviamo ora \(B\) in simboli:
    \[
        B\colon (\forall n \in \N) (n \text{ è dispari}),
    \]
    dove \(P(n)\colon n \text{ è dispari}\). Notiamo che \(B\) è Falsa: esiste almeno un numero pari che appartiene ad \(\N\). Sarà allora Vera \(\neg B\):
    \begin{alignat*}{1}
        \neg B\colon (\exists n \in \N)\; \neg \left(P(n)\right)%
            &\implies (\exists n \in \N)\; \neg (n \text{ è dispari}) \\
            &\implies (\exists n \in \N)\; (n \text{ non è dispari}) \\
            &\implies (\exists n \in \N)\; (n \text{ è pari}),
    \end{alignat*}
    come volevasi dimostrare.
\end{example}

\begin{example}
    Si consideri \(C\): ``esiste un numero naturale che è un quadrato perfetto''. Un quadrato è perfetto se la sua radice è numero intero. Scriviamo \(C\) in simboli:
    \[
        C\colon (\exists n \in \N) (\exists h \in \N \tc n = h^2),
    \]
    dove \(P(n)\colon (\exists h \in \N \tc n = h^2)\). Questa proposizione è Vera: esiste, ad esempio \(4\) la cui radice \(\sqrt{4} = 2\) verifica la relazione precedente.
    Logicamente \(\neg C\) sarà Falsa:
    \begin{alignat*}{1}
        \neg C\colon (\forall n \in \N) \big(\neg P(n)\big)%
            &\implies (\forall n \in \N)\; \neg (\exists h \in \N \tc n = h^2) \\
            &\implies (\forall n \in \N) \big(\forall h \in \N\; \neg (n = h^2)\big) \\
            &\implies (\forall n \in \N) (\forall h \in \N n \neq h^2).
    \end{alignat*}
\end{example}

\noindent In generale, le negazioni dei predicati col quantificatore universale ``per ogni'' sono le seguenti:
\[
    \neg \left((\forall x \in U_x)\left(P(x)\right)\right)
    \implies
    (\exists x \in U_x)\left(\neg P(x)\right);
\]
mentre per l'altro quantificatore universale ``esiste'' sono:
\[
    \neg \left((\exists x \in U_x)\left(P(x)\right)\right)
    \implies
    (\forall x \in U_x)\left(\neg P(x)\right).
\]

Segue ora una lista di esempi sui predicati e le loro relative negazioni.
\begin{examples}
    Sia \(U\) l'insieme di tutti gli esseri umani.
    \begin{enumerate}
        \item \(P_1 \colon \text{ tutti hanno almeno un cugino}\), è Falsa. In simboli:
            \begin{alignat*}{2}
                P_1\colon
                    &(\forall x \in U) (\exists y \in U \tc y \text{ è cugino di } x)
                        & \quad 0                                                       \\
                \neg P_1\colon
                    &(\exists x \in U) (\forall y \in U \; y \text{ non è cugino di } x)
                        & \quad 1
            \end{alignat*}
%
        \item \(P_2 \colon \text{ tutti gli esseri umani sono cugini tra di loro}\), è Falsa. In simboli:
            \begin{alignat*}{2}
                P_2 \colon
                    &(\forall x,y \in U) (x \text{ è cugino di } y)
                        & \quad 0                                                       \\
                \neg P_2 \colon
                    &(\exists x,y \in U) (x \text{ non è cugino di } y)
                        & \quad 1
            \end{alignat*}
%
        \item \(P_3 \colon 3 \text{ è un numero primo}\), che è Vera e \(P_4 \colon \sqrt{5} \text{ è un numero razionale}\), che è Falsa.
            \begin{alignat*}{3}
                (P_3 \land P_4), \quad &\neg(P_3 \lor P_4), \quad &(\neg P_3 \land \neg P_4) &\quad 0 \\
                \neg(P_3 \land P_4), \quad &(\neg P_3 \lor \neg P_4), \quad &(P_3 \lor P_4)  &\quad 1
            \end{alignat*}
%
        \item \(P_5 \colon (\forall a \in \N) \left((\exists y \in \Z) (y - a^2 = -1)\right)\), è Vera.
            \begin{alignat*}{2}
                P_5 \colon
                    &(\forall a \in \N) \left((\exists y \in \Z) (y = a^2 -1)\right)
                        &\quad 1 \\
                \neg P_5 \colon
                    &(\exists a \in \N) \left((\forall y \in \Z) (y \neq a^2 - 1)\right)
                        &\quad 0
            \end{alignat*}
    \end{enumerate}
\end{examples}

\begin{remark}
    Considerato il predicato
    \[
        (\forall x \in U) \left(P(x)\right)
    \]
    si deduce che l'universale deduce il particolare:
    \[
        (\exists x \in U) \left(P(x)\right)
    \]
    che risulta essere Vera. Se \(P(x)\) è Vero per ogni \(x\), lo sarà anche per una singola \(x\). Non è Vero l'inverso, infatti:
    \[
        (\exists x \in U) \left(P(x)\right) \centernot\longrightarrow (\forall x \in U) \left(P(x)\right);
    \]
    poiché se \(P(x)\) è Vero per una \(x\), non è detto che lo sia per tutte le altre.
\end{remark}
