% !TEX root = ../Matematica discreta.tex

\begin{figure}[h]
\centering
    \begin{tikzpicture}
        \node (A) at (-2.5, 1.5) {\(A\)};
        \node (B) at (1, 0)      {\(B\)};

        \draw
        [
            pattern = north west lines,
            even odd rule
        ]
        (0, 0) ellipse (3cm and 2cm)
        (1, 0) circle  (1cm);

        \node[preaction={fill, white}, rounded corners=3mm] (C-ab) at (-1, 0.5) {\(\complement_A(B)\)};
    \end{tikzpicture}
\caption{Insieme \(\complement_A(B)\), complementare di \(B\) rispetto ad \(A\)}\label{fig:insieme_complementare}
\end{figure}