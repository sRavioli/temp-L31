% !TEX root = ../Matematica discreta.tex

\begin{figure}[h]
\centering
    \begin{tikzpicture}

        \node (A)   at (-2.7, +1)  {\(A\)};
        \node (B)   at (+2.7, +1)  {\(B\)};

        \draw (+1, 0) ellipse (2cm and 1cm);
        \draw (-1, 0) ellipse (2cm and 1cm);

        \clip (+1, 0) ellipse (2cm and 1cm);
        \fill[dashed, pattern=north west lines] (-1, 0) ellipse (2cm and 1cm);

        \node[preaction={fill, white}, rounded corners=3mm] (AuB) at (0.0, 0.0) {\(A \cap B\)};
    \end{tikzpicture}
\caption{Rappresentazione dell'insieme intersezione}\label{fig:insieme_intersezione}
\end{figure}